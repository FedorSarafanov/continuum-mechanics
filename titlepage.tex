%!TEX root = lections.tex
\begin{titlepage}
\thispagestyle{empty}

\begin{center}
	{\small\textsc{Нижегородский государственный университет имени Н.\,И. Лобачевского}}
	\vskip 3pt \hrule \vskip 5pt
	{\small\textsc{Радиофизический факультет}}

	\vfill

	\begin{spacing}{2}
	% {\huge \bf  Курин\, В.В.}\\[1.5em]
	{\Huge \bf  Лекции по механике сплошных сред}\\%\vspace{1em}
	\end{spacing}
	\vspace{1em}
	{ Набор и вёрстка:}\\[.5em]
	{ 
		\href{https://github.com/fedorsarafanov}{\color{black}{Сарафанов Ф.Г.}}, 
		\href{https://github.com/greengrocer98}{\color{black}{Есюнин М.В.}},
	\\ Платонова М.В.}\\[2em]
	% {\large }\\
	\vspace{1em}
\end{center}

\textbf{Disclaimer.} В данном документе нами набраны лекции по механике сплошных сред, прочитанные на 3 курсе радиофизического факультета ННГУ \textbf{Сергеем Николаевичем Гурбатовым}. Документ призван облегчить подготовку к зачетам и экзаменам и восполнить пробелы в знаниях читателя по механике сплошных сред. Разрешено копирование и распространение данного документа с обязательным указанием первоисточника. 

При обнаружении ошибок, опечаток и прочих вещей, требующих исправления, можно либо создать issues в \href{https://github.com/FedorSarafanov/continuum-mechanics}{репозитории на github.com}, либо написать по электронной почте  \href{mailto:sfg180@yandex.ru}{\color{black}{sfg180@yandex.ru}}.

\begin{center}
	\vfill
	2 февраля -- \today\\Нижний Новгород
\end{center}

\end{titlepage}