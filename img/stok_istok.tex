\documentclass[tikz]{standalone}

\usepackage{cmap}
\usepackage[T2A]{fontenc}
\usepackage[utf8x]{inputenc}
\usepackage[english, russian]{babel}
\usepackage{tikz}
\usetikzlibrary{quotes,angles,calc,decorations.markings}
% GNUPL
% Draw line annotation
% Input:
%   #1 Line offset (optional)
%   #2 Line angle
%   #3 Line length
%   #5 Line label
% Example:
%   \lineann[1]{30}{2}{$L_1$}
\newcommand{\lineann}[4][0.5]{%
    \begin{scope}[rotate=#2, blue,inner sep=2pt]
        \draw[densely dashed, blue!40] (0,0) -- +(0,#1)
            node [coordinate, near end] (a) {};
        \draw[densely dashed, blue!40] (#3,0) -- +(0,#1)
            node [coordinate, near end] (b) {};
        \draw[|<->|] (a) -- node[fill=white] {#4} (b);
    \end{scope}
}
\begin{document}
\pagestyle{empty}


\begin{tikzpicture}[
    scale=1.5, 
    arrowline/.style={black!60},
    kos/.style={black},
    tangent/.style={
        decoration={
            markings,% switch on markings
            mark=
                at position #1
                with
                {
                    \coordinate (tangent point-\pgfkeysvalueof{/pgf/decoration/mark info/sequence number}) at (0pt,0pt);
                    \coordinate (tangent unit vector-\pgfkeysvalueof{/pgf/decoration/mark info/sequence number}) at (1,0pt);
                    \coordinate (tangent orthogonal unit vector-\pgfkeysvalueof{/pgf/decoration/mark info/sequence number}) at (0pt,1);
                }
        },
        postaction=decorate
    },
    use tangent/.style={
        shift=(tangent point-#1),
        x=(tangent unit vector-#1),
        y=(tangent orthogonal unit vector-#1)
    },
    use tangent/.default=1
    ]
    \foreach \ang in {45,90,...,360}{
    \draw[] (0,0)  -- node[rotate=\ang,pos=0.5]{\tikz{\draw[->](0,0) -- ++(0.1,0)}} ++(\ang:0.5);
    }
    \draw (0,0.5) node[above] {Исток};

    \begin{scope}[xshift=1.25cm]
            \foreach \ang in {45,90,...,360}{
    \draw[] (0,0)  -- node[rotate=\ang,pos=0.5]{\tikz{\draw[<-](0,0) -- ++(0.1,0)}} ++(\ang:0.5);
    }
    \draw (0,0.5) node[above] {Сток};
    \end{scope}
    % \draw[tangent=0.6,tangent=0,arrowline] (0,0.8)  to [out=0,in=120] node[sloped,pos=0.5]{\tikz{\draw[->,arrowline](0,0) -- ++(0.1,0)}} (3,1.2);

    % \draw [use tangent=1,->,kos] (0,0) -- (0.75,0) node[right] {$\vec{v}$};
    % \draw [use tangent=2,->,kos] (0,0) -- (0.75,0) node[right] {$\vec{v}$};
    % \draw[tangent=0.79,arrowline] (0,0)  to [out=0,in=120] node[sloped,pos=0.5]{\tikz{\draw[->,arrowline](0,0) -- ++(0.1,0)}} (3,0);
    % \draw [use tangent,->,kos] (0,0) -- (0.75,0) node[right] {$\vec{v}$};

\end{tikzpicture}


\end{document}