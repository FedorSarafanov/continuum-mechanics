\documentclass[tikz]{standalone}

\usepackage{cmap}
\usepackage[T2A]{fontenc}
\usepackage[utf8x]{inputenc}
\usepackage[english, russian]{babel}
\usepackage{tikz}
\colorlet{myblue}{cyan!40!white}
\usetikzlibrary{quotes,angles,calc,decorations.markings}
\usepackage[outline]{contour}
\contourlength{1.4pt} 
% GNUPL
% Draw line annotation
% Input:
%   #1 Line offset (optional)
%   #2 Line angle
%   #3 Line length
%   #5 Line label
% Example:
%   \lineann[1]{30}{2}{$L_1$}
\newcommand{\lineann}[4][0.5]{%
    \begin{scope}[rotate=#2, blue,inner sep=2pt]
        \draw[densely dashed, blue!40] (0,0) -- +(0,#1)
            node [coordinate, near end] (a) {};
        \draw[densely dashed, blue!40] (#3,0) -- +(0,#1)
            node [coordinate, near end] (b) {};
        \draw[|<->|] (a) -- node[fill=white] {#4} (b);
    \end{scope}
}

\tikzset{
  pics/carc/.style args={#1:#2:#3}{
    code={
      \draw[pic actions] (#1:#3) arc(#1:#2:#3);
    }
  }
}


\begin{document}
\pagestyle{empty}


\begin{tikzpicture}[
    scale=1, 
    arrowline/.style={black!60},
    kos/.style={black},
    tangent/.style={
        decoration={
            markings,% switch on markings
            mark=
                at position #1
                with
                {
                    \coordinate (tangent point-\pgfkeysvalueof{/pgf/decoration/mark info/sequence number}) at (0pt,0pt);
                    \coordinate (tangent unit vector-\pgfkeysvalueof{/pgf/decoration/mark info/sequence number}) at (1,0pt);
                    \coordinate (tangent orthogonal unit vector-\pgfkeysvalueof{/pgf/decoration/mark info/sequence number}) at (0pt,1);
                }
        },
        postaction=decorate
    },
    use tangent/.style={
        shift=(tangent point-#1),
        x=(tangent unit vector-#1),
        y=(tangent orthogonal unit vector-#1)
    },
    use tangent/.default=1
    ]
    % \foreach \y in {-0.5,-0.25,0,0.25,0.5}{
    % \draw[-latex] (0,\y) -- ++ (1,0);
    % }
    % \draw (1,0.5) node[right] {$\vec{v}$};

    % \draw (0,0) node[below=0.2em] {$\vec{\Omega}$};
  %   \draw (2.5,-0.5) node [below] {Поток};
  %   \foreach \y in {-0.4,-0.2,0,0.2,0.4}{
  %   \draw[white] (-3,\y) -- ++ (1,0);
  %   }

  % \draw[thick] (0,0) pic[black, -latex]{carc=-20:20:1.3cm};
    % \foreach \r in {0.2,0.4,...,0.8}{
    % \draw[decoration={markings, mark=at position 0 with {\arrow{<}}},
    %     postaction={decorate}] (0,0) circle (\r);
    % }

    % \draw[->] (-2,0) -- (2,0) node[right] {$x$};
    % \draw[->] (0,-2) -- (0,2) node[above] {$y$};

    % \draw[decoration={markings, mark=at position 0 with {\arrow{>}}},
    %      postaction={decorate}] (0,0.8) circle (0.8);
    % \draw[decoration={markings, mark=at position 0 with {\arrow{>}}},
    %      postaction={decorate}] (0,0.5) circle (0.5);

    % \draw[decoration={markings, mark=at position 0 with {\arrow{<}}},
    %      postaction={decorate}] (0,-0.8) circle (0.8);
    % \draw[decoration={markings, mark=at position 0 with {\arrow{<}}},
    %      postaction={decorate}] (0,-0.5) circle (0.5);

    % \draw[densely dashed] (0.9,0) circle (0.9);
    % \draw[densely dashed] (-0.9,0) circle (0.9);

    % \draw[fill=black!10] (0,0) circle (1);
    % \fill[black] (0,0) circle (1.2pt);
    % \draw[->] (0,0) -- node {\contour{white}{$R$}} ++(60:1);
    % \foreach \x in {0.2,0.4,...,1.8}{
    % \draw[->] (\x,-0.2) -- ++(0,1.4);
    % }
    % \draw (2,0.5) node[right] {$\varphi=\mathrm{const}$};
    % \draw (0,0.5) node[left,white] {$\varphi=\mathrm{const}$};
    % \draw (1,1.2) node[above] {$\Psi=\mathrm{const}$};
    % \draw[tangent=0.6,tangent=0,arrowline] (0,0.8)  to [out=0,in=120] node[sloped,pos=0.5]{\tikz{\draw[->,arrowline](0,0) -- ++(0.1,0)}} (3,1.2);

    % \draw [use tangent=1,->,kos] (0,0) -- (0.75,0) node[right] {$\vec{v}$};
    % \draw [use tangent=2,->,kos] (0,0) -- (0.75,0) node[right] {$\vec{v}$};
    % \draw[tangent=0.79,arrowline] (0,0)  to [out=0,in=120] node[sloped,pos=0.5]{\tikz{\draw[->,arrowline](0,0) -- ++(0.1,0)}} (3,0);
    % \draw [use tangent,->,kos] (0,0) -- (0.75,0) node[right] {$\vec{v}$};


    % \draw[->] (0,0) -- ++(-30:1) node[right] {$r$};
    % \draw[draw=none] (0,0) -- ++(180+30:1) node[left,white] {$r$};

    % \fill[myblue] (-0.7,0) rectangle (0.7,2);
    % \draw[fill=myblue, densely dashed] (0,0) ellipse (0.7 and 0.3);
    % \draw[fill=myblue] (0,2) ellipse (0.7 and 0.3);
    % \draw (0.7,0) -- ++ (0,2);
    % \draw (-0.7,0) -- ++ (0,2);
    % \begin{scope}
    % \clip (-0.7,0) rectangle (0.7,-0.4);
    % \draw[] (0,0) ellipse (0.7 and 0.3);    
    % \end{scope}

    % \draw[->] (0,2) -- (0,2.5) node[above] {$z$};
    % \draw[densely dashed] (0,2) -- (0,0) -- ++ (-30:1);

    % \draw[-latex] (0.35,2) -- ++ (0,0.5) node[right] {$\vec{\Omega}$};


    % % \draw (0,0) node[scale=0.7] {$\bigodot$} node[left=0.2em] {\contour{white}{$z$}} [fill=white, draw=none] circle (3.2pt);

    \draw[->] (-0.2,0) -- (3,0) node[right] {$r$};
    \draw[->] (0,-0.2) -- (0,2) node[above] {$p$};
    \draw[domain=1:2.7,smooth, samples=100,variable=\r,blue] plot 
        ({\r},{1.5-1/2/\r^2}); 

    \draw[domain=0:1,smooth,variable=\r,blue] plot 
        ({\r},{1.5-1+\r^2/2});

    \draw[densely dashed] (1,0) node[below] {$R$} -- (1,1);
    \draw[densely dashed] (0,1.5) node[left] {$p_0$} -- ++(2.7,0);
\end{tikzpicture}


\end{document}