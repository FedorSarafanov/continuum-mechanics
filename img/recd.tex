\documentclass{standalone}
\usepackage{pgfplots}
\usepackage[T2A]{fontenc}
\usepackage[utf8x]{inputenc}
\usepackage[russian]{babel}
% \usepackage[utf8x]{inputenc}
\usepackage{amsmath}
\usepackage{amssymb,physics}
\pgfplotsset{compat=newest}
\usetikzlibrary{decorations.markings}
\tikzset{->-/.style={decoration={
  markings,
  mark=at position #1 with {\arrow{stealth}}},postaction={decorate}}}
  \tikzset{-<-/.style={decoration={
  markings,
  mark=at position #1 with {\arrow{<}}},postaction={decorate}}}
\begin{document}
\begin{tikzpicture}
  \begin{axis}[
      %%%%%%%%%%%%%%%%%%%%%%%%%%%%% НАСТРОЙКИ ГРАФИКА %%%%%%%%%%%%%%%%%%%%%%%%%%%%%
      height=5cm,
      % width=10.5cm,
      scale=2.3,
      % grid=both,        % вКлючаем отоБражение сетКи на графиКе
      %
      xlabel={$\mathrm{Re}$},       % ПодПись оси X
      ylabel={$C_d$},       % ПодПись оси Y
      %
      major grid style={
        line width=0.5pt,   % толщина основных линий сетКи
        % draw=black!50,    % цвет основных линий сетКи: 50% черного (80% Белого) 
        % draw=none,
      },
      %
      minor grid style={
        line width=0.5pt,   % толщина Промежуточных линий сетКи
        % draw=black!20,    % цвет Промежуточных линий сетКи
        % draw=none,
      },
      %
      % minor x tick num=0,   % Количество Промежуточных линий между основными
      % minor y tick num=0,   % Количество Промежуточных линий между основными
      %
      ticklabel style={
        scale=1      % уменьшим размер ПодПисей метоК на осях
      },    
      %
      % ticks=none,
        axis lines=left,    % выравнивание оси y:  middle (в нуле)|left|right
        %
      % ymin = 7,% маКсимумы и минимумы осей на графиКе
      % ymax = 10,
      % ymax = 0.28,
      % ymin = 0,
      % ymax= 1.5,
        % x label style={
        % at={(current axis.right of origin)}, 
        %     xshift=1ex, anchor=center
        % },
      % enlargelimits=true,
      ymin = 0, 
      xmin=-0.001,
      % ymax = 3.5,
      % xmin=-3*pi/2*0.8,
      % xmax=2*pi,
      % xmin=-2*pi,
      % xmax=7.5,
      % ymin=-2.1,
      % ymax=0.5,
      % ymin=-2.25,
       % clip=false,
      restrict y to domain=0:200,
      restrict x to domain=0:200,
      unbounded coords=jump,
      %
      % xtick distance=1,   % расстояние между метКами По оси X
      ytick distance=1/4,    % расстояние между метКами По оси Y
        xticklabels={},
        yticklabels={{},,,,,,1,,,,2,,,,3}, 
        xtick=\empty,
        % ytick=\empty,
      % disabledatascaling,
      % ymin={-11/16},
      % extra y ticks={-9/16,-6/16},  % доПолнительные метКи на осях
      extra x ticks={0,1,2,3,4,5,6,7,8,9},  % доПолнительные метКи на осях
      extra x tick labels={ % можно уКазать сПециальные ПодПиси К ним
        {$1\vphantom{10^0}$},
        {$10^1$},
        {$10^2$},
        {$10^3$},
        {$10^4$},
        {$10^5$},
        {$10^6$},
        {$10^7$},
          % {$\delta$},
      %     {$\frac{\pi}{a}$}
        },    
      extra y ticks={0},  % доПолнительные метКи на осях
      extra y tick labels={ % можно уКазать сПециальные ПодПиси К ним
          {$\mathrm{0}$}
        },    
      extra y tick style={
        % yshift=1em,
        % black,
        % fill=white
      },
      unit vector ratio = 0.8 1,% масштаБ 1:1 осей X и Y
      % x={(1cm/1.3,0cm)}, y={(0cm,50cm/1.3)},
      %
      % x axis line style ={draw=none},
      % x axis line style ={d},
       % xmin=0,
       % ymin=0,
       % xmax=1,
       ymax=3.5,
      x axis line style ={->},
      y axis line style ={->},
        % y tick/.style={
        %   draw=none,
        %   % semithick,
        % },
      % x label style={
      %   xshift=0.5em,
      %   at={(axis description cs:1.05,0)},
      %   anchor=center,    % расПоложение метКи ровно в точКе (1.1,0)
      %   % rotate=360,     % вооБще метКу еще можно Повернуть)
      %   % black       % цвет метКи
      % },
      %   y label style={
      %     at={(axis cs:0.2,3.7)},
      %     yshift=1em,
      %     rotate=-90,
      %     anchor=center,    % расПоложение метКи ровно в точКе (0,1.1)
      %     black       % цвет метКи
      %   },      
      %             
      %%%%%%%%%%%%%%%%%%%%%%%%%%%%%%%%%%%%%%%%%%%%%%%%%%%%%%%%%s%%%%%%%%%%%%%%%%%%%%
      % stressstrainset
    ]

    % Colored background
    % \addplot graphics[xmin = 0, xmax = 2, ymin = 0, ymax = 1] {myplot.png};
    % Contour lines
    % \draw (axis cs:0,0) circle [blue, radius=1];
        \addplot[smooth,blue,semithick] table[col sep=tab]{data/recd.tsv} node[pos=0.5] {};
        % \addplot[mark=none,postaction={decorate}] table[col sep=comma] {myplot.csv};

        \draw[densely dashed, gray] (1.3,0) -- (1.3,5);
        \draw[densely dashed, gray] (2.16,0) -- (2.16,5);
        \draw[densely dashed, gray] (5.18,0) -- (5.18,5);

        \draw[draw=none] ({1.3/2},0) -- ({1.3/2},3) node[sloped,pos=0.25,scale=1,align=center]{Стационарный\\ $\pdv{v}{t}=0$};

        \draw[draw=none] ({(1.3+2.16)/2},0) -- ({(1.3+2.16)/2},3) node[sloped,pos=0.25,scale=1,align=center]{Периодический\\ ламинарный};
        \draw[draw=none] ({(5.18+2.16)/2},0) -- ({(5.18+2.16)/2},3) node[pos=0.17,scale=1,align=center]{Периодический\\ турбулентный};

        % \addplot3[blue,
        %     quiver={
        %      u={-sin(pi*x/2*180/pi)*cos(pi*y*180/pi)},
        %      v={-cos(pi*x/2*180/pi)*sin(pi*y*180/pi)},
        %      scale arrows=0.1,
        %     },
        %     densely dashed,
        %     -stealth,samples=15]
        %         {0};

                % \draw[fill=white, draw=none] (0.9,0.5) rectangle (1.1,0.6);
    % \draw[dashed] (0.1,0.1) to[out=90-19, in=-90+19] (0.1,0.9);
    % \draw[dashed] (0.3,0.1) to[out=90-32, in=-90+32] (0.3,0.9);

    % \draw[dashed] (0.1,0.1) to[out=90-19, in=-90+19] (0.1,0.9);
    % \draw[dashed] (0.3,0.1) to[out=90-32, in=-90+32] (0.3,0.9);      
% \draw[dashed] (0.20,0.90) .. controls (0.50,0.75) and (0.50,0.25) .. (0.20,0.10);
% \draw[dashed] (1.80,0.90) .. controls (1.55,0.75) and (1.55,0.25) .. (1.80,0.10);
% \draw[dashed] (1.70,0.05) .. controls (1.20,0.25) and (0.80,0.25) .. (0.30,0.05);
% \draw[dashed] (0.30,0.95) .. controls (0.80,0.75) and (1.20,0.75) .. (1.70,0.95);
% \draw[dashed] (0.15,0.90) .. controls (0.30,0.69) and (0.30,0.31) .. (0.15,0.10);
% \draw[dashed] (1.85,0.90) .. controls (1.70,0.69) and (1.70,0.31) .. (1.85,0.10);
  \end{axis}
\end{tikzpicture}
\end{document}
