%!TEX root = ../main.tex
\subsection{Теорема о сохранении циркуляции скорости – теорема Томсона. Понятие о потенциальных и вихревых движениях жидкости.}

Введём понятие циркуляции скорости – интеграл, взятый вдоль некоторого замкнутого контура
\begin{align*}
\Gamma = \oint \limits_ { L } \vec { v } d \vec { r }
\end{align*}
Докажем теорему о сохранении циркуляции скорости – теорему Томсона (лорда Кельвина):
\textbf{Циркуляция скорости вдоль замкнутого контура, перемещающего в идеальной жидкости, остается постоянной.}
\begin{figure}[h]
	\centering
	\includegraphics[scale=1]{example-image-c}
	\caption{два контура:начальный и смещенный}
	\label{fig:figure13}
\end{figure}
Выберем замкнутый контур, состоящий из фиксированных частиц («жидкий» контур) и перемещающийся вместе с ними.  Найдем полную производную по времени от этого контура. Происходит изменение как скорости, так и изменение контура во времени
\begin{align*}
\frac{d \Gamma}{d t}=\frac{d}{d t} \oint\limits_{L} \vec{v} d \vec{r}=\oint\limits_{L} \frac{d \vec{v}}{d t} d \vec{r}+\oint\limits_L \vec{v} d\left(\frac{d \vec{r}}{d t}\right)
\end{align*}

Используем определение скорости и уравнение Эйлера
\begin{align*}
\frac { \vec { d v } } { d t } = - \frac { \nabla p } { \rho } - \vec { f }
\end{align*}

Пусть внешняя сила потенциальна, а процесс адиабатический
\begin{align*}
&{ \vec { f } = - \nabla u } \\ 
&{ \frac { \nabla p } { \rho } = \nabla W }
\end{align*}
Здесь $W$ энтальпия.  Учтем, что $ ( \nabla \phi d \vec { r } ) = d \phi $ и окончательно получим
\begin{align*}
&\frac { d \Gamma } { d t } = \oint _ { l } d \left( \frac { v ^ { 2 } } { 2 } - W - u \right) = 0 \\
& \Gamma = const
\end{align*}

\begin{itemize}
	\item Следствие 1
	Используем теорему Стокса
	\begin{align*}
	\oint \limits_ { L } \vec { v } d \vec { r } = \int \limits_ { S } \vec { n } \operatorname { rot } \vec { v } d S
	\end{align*}
	\begin{figure}[h]
		\centering
		\includegraphics[scale=1]{example-image-c}
		\caption{Контур и поверхность, натянутая на этот контур}
		\label{fig:figure14}
	\end{figure}
	Для потенциальных течений $ \operatorname{r o t} \vec { v } = 0 , \Gamma = 0$
	Циркуляция скорости по \textbf{односвязанному контуру} в потенциальном течении идеальной жидкости равна нулю.
	\item Следствие 2
	\begin{align*}
	\Gamma = \oint \limits_ { L } \vec { v } d \vec { r } = \int \limits_ { S } \vec { n } \operatorname{rot}\vec { v } d S = const
	\end{align*}
	Поток вихря через поверхность, натянутую на  \textbf{односвязанный контур} в потенциальном течении идеальной жидкости величина постоянная. 
	\item Следствие 3

	В потенциальном течении не может быть \textbf{замкнутых линий тока} (иначе, взяв ее в качестве контура мы получим, что циркуляция вдоль данного контура не равна нулю).
	\item Следствие 4

	В однородной несжимаемой жидкости можно исключить из рассмотрения уравнений движения давление.  Запишем уравнение Эйлера в форме Громэко-Лэмба
	\begin{align*}
	\frac { \partial \vec { v } } { \partial t } + \nabla \left( \frac { v ^ { 2 } } { 2 } \right) - [ \vec { v } \operatorname{r o t} \vec { v } ] = \nabla ( W + u )
	\end{align*}
	Возьмем от него ротор и учтем, что ротор от градиента равен нулю ($ \operatorname { rot } \nabla \vec { v } = 0 $)
	\begin{align*}
	&{ \frac { \partial } { \partial t } \operatorname{r o t} \vec { v } = \operatorname{r o t} [ \vec { v } \operatorname{r o t} \vec { v } ] } \\
	&{\Div \vec { v } = 0 }
	\end{align*}
	Полное описание поля скорости с помощью одного уравнения.
\end{itemize}
\subsubsection{Основные выводы из теоремы Томсона}
Если в какой-то точке линии тока завихренность отсутствует, то она отсутствует и вдоль этой линии.

На первый взгляд отсюда следует:
\begin{enumerate}
	\item Стационарное обтекание любого тела набегающим из бесконечности потоком должно быть потенциальным $\vec { v } = \text { const } ,\operatorname { rot } \vec { v } = 0$
	\item Если движение жидкости потенциально в некоторый момент времени, то оно будет потенциальным и в дальнейшем. В частности:

	Потенциальным должно быть всякое движение, при котором в начальный момент жидкость покоилась. В реалии, однако, это этот имеет ограниченную область применимости. Дело в том, что приведенное выше утверждение о сохранении ротора скорости вдоль линии тока неприменимо для линий проходящих воль поверхности твердого тела. Около стенки \textbf{нельзя провести односвязный замкнутый контур}. 

	Уравнения движения идеальной жидкости допускают решения в которых на поверхности твердого тела, обтекаемого жидкостью твердого тела происходит «отрыв» струи: линии тока, следовавшие вдоль поверхности, в некотором месте отрываются от него, уходя в глубь жидкости. Возникает застойная область и на границе течение становится непотенциальным 
	\begin{figure}[H]
		\centering
		\includegraphics[scale=1]{example-image-c}
		\caption{Обтекание тела с застойными зонами}
		\label{fig:figure14}
	\end{figure}
	Возникает поверхность «тангенциального» разрыва. Скорость терпит разрыв непрерывности.


	При учете таких разрывных решений решение уравнений идеальной жидкости неоднозначно: наряду с непрерывным решением появляется бесконечнок множество разрывных решений. При этом разрывные решения не имеют физического смысла: так как тангенциальные разрывы \textbf{абсолютно неустойчивы}, в результате чего движение жидкости становится \textbf{турбулентным}.

	Реальное течение безусловно однозначно. Всякая жидкость обладает вязкостью. Малая вязкость практически не проявляется во всем пространстве, но она будет играть определяющую роль в пристеночной области (пограничный слой).

	\textbf{Тем не менее в ряде случаев это достаточно хорошее приближение}
	\begin{itemize}
		\item Хорошо обтекаемые тела (самолет) автомобиль, корабль) – движение жидкости от потенциального отличатся только в области «пограничного» слоя и «следа» позади тела.
		\item Нестационарные малые колебания
		\begin{figure}[H]
			\centering
			\includegraphics[scale=1]{example-image-c}
			\caption{Сфера размером $l$ и сфера, смешенная на $a$.}
			\label{fig:figure15}
		\end{figure}
		$l$ - линейный размер тела, $a$ - характерная амплитуда колебаний, $v$ - скорость колеблющегося тела. Если $l \gg a$, то движение жидкости вокруг тела потенциально. Оценим порядок величины различных членов в уравнении Эйлера
		\begin{align*}
		\frac { \partial \vec { v } } { \partial t } + ( \vec { v } \nabla ) \vec { v } = - \nabla W
		\end{align*}
		Характерные масштабы изменения скорости $v$ порядка $l$, а изменения скорости во времени определятся частотой колебаний. Оценка членов в уравнении Эйлера
		\begin{align*}
		\left| \frac { \partial \vec { v } } { \partial t } \right| / \left| \vec { v } \nabla ) \vec { v } \right| \propto \frac { u ^ { 2 } } { a } / \frac { l } { u ^ { 2 } } \propto \frac { l } { a } \gg 1
		\end{align*}
		То есть уравнение Эйлера имеет вид:
		\begin{align*}
		\frac { \partial \vec { v } } { \partial t } = - \nabla W
		\end{align*}
		Взяв ротот, имеем:
		\begin{align*}
		\frac { \partial } { \partial t } \operatorname{rot} \vec { v } = 0 , \quad \operatorname { rot } \vec { v } = \operatorname { const } = 0
		\end{align*}
		Так как при колебательном движении среднее значение по периоду равно нулю.

		Таким образом нестационарные \textbf{малые колебания потенциальны}.
	\end{itemize}
\end{enumerate}
\subsection{Потенциальные течения несжимаемой жидкости. Парадокс Даламбера. Присоединенная масса.}

В идеальной баротропной жидкости в поле потенциальных сил вихри не исчезают и не возникают. Если в начальный момент течение было потенциально, то оно будет потенциальным всегда. В ряде случаев это достаточно хорошее приближение, а уравнения гидродинамики существенно упрощаются. 
\subsubsection{Уравнения гидродинамики несжимаемой идеальной жидкости, когда движение потенциально}
Уравнение непрерывности несжимаемой жидкости:
\begin{align*}
\frac { d \rho } { d t } = 0 ,\quad \Div ( \vec { v } ) = 0
\end{align*}
Поле скорости несжимаемой жидкости соленоидально. Уравнение Эйлера для несжимаемой жидкости:
\begin{align*}
\frac { \partial \vec { v } } { \partial t } + ( \vec { v } \nabla ) \vec { v } = - \nabla \frac { p } { \rho } + \vec { g }, \quad \vec { g } = - g \nabla z
\end{align*}
Следовательно
\begin{align*}
\frac { \partial } { \partial t } \operatorname { rot } \vec { v } = \operatorname { rot } [ \vec{v} \operatorname { rot } \vec{v}  ]
\end{align*}
то есть движение потенциально. Если сначала движение было потенциально, то и в дальнейшем оно будет потенциальным. Поэтому можно ввести потенциал поля скорости
\begin{align*}
 &\operatorname { rot } \vec { v } = 0 , \quad  { \vec { v } = \operatorname { grad } \varphi } \\  
 &\operatorname { divgrad } \varphi = \Delta \varphi = 0 
\end{align*}
Таким образом, описание потенциального движения идеальной несжимаемой жидкости описывается уравнением Лапласа:
\begin{align*}
\Delta \varphi = 0 , \quad \vec { v } = \operatorname { grad } \varphi
\end{align*}
Для решения этого уравнения также необходимы граничные условия.

\textbf{Условие на протекание}: нормальная компонента скорости жидкости на поверхности тела должна совпадать с проекцией $v_n=\frac { \partial \varphi } { \partial n }=u_n$ скорости самого тела на эту нормаль. \textbf{Второе условие}: обычно используют значение потенциала на бесконечности.

Возникает вопрос: \textbf{как найти давление?}. Ответ: из уравнения Бернулли.
\begin{align*}
& \frac { \partial \varphi } { \partial t } + \frac { v ^ { 2 } } { 2 } + \frac { p } { \rho } + u = \operatorname { const } \\
& p = - \left( \frac { \partial \varphi } { \partial t } + \frac { v ^ { 2 } } { 2 } + u \right) + \operatorname { const }
\end{align*}
Константа может зависеть от времени. Для стационарного течения:
\begin{align*}
p = - \left( \frac { v ^ { 2 } } { 2 } + u \right) + \operatorname { const }
\end{align*}

Рассмотрим несколько частных решений уравнения Лапласа(вспомним электростатику)
\begin{itemize}
	\item Пример 1.
	Сдвиговый поток (поле плоского конденсатора), все частицы жидкости двигаются с постоянной скоростью

\end{itemize}

\begin{equation}
    x^{-1}
    x^2

\end{equation}
