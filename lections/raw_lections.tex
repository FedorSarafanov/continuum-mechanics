%!TEX root = ../lections.tex
\begin{center}
	\textbf{======Здесь помер Максим=======}
\end{center}
Теория аналитических функций задач гидродинамики плоских течений.

У нас есть некоторое комплексное число
\begin{equation}
	z=x+iy 
	\quad \leftrightarrow \quad
	F(z)=\alpha+i \beta
\end{equation}
Функция аналитическая, если независимо от направления стремления $\Delta z$ к нулю существует предел:
\begin{equation}
	F'(z)=\lim\limits_{\Delta z\to0}\frac{F(z+\Delta z)-F(z)}{\Delta z}
\end{equation}

Пример:
\begin{equation}
	F'(z)=\lim\limits_{\Delta x\to0}\frac{\alpha(x+\Delta x,y)+i\beta(x+\Delta x,y)-\alpha(x,y)-i\beta(x,y)}{\Delta x}=
		\pdv{\alpha}{x}+i\pdv{\beta}{x}
\end{equation}

Второй пример:
\begin{equation}
	\Delta z=i\Delta y
\end{equation}
\begin{equation}
	F'(z)=\lim\limits_{\Delta y\to0}\frac{\alpha(x,y+\Delta y)+i\beta(x,y+\Delta y)-\alpha(x,y)-i\beta(x,y)}{i\Delta y}=
		\pdv{\alpha}{y}-\pdv{\alpha}{y}
\end{equation}

Итак, условие аналитичности
\begin{gather}
	\label{eq2}
	\pdv{\alpha}{x}=\pdv{\beta}{y}\\
	\pdv{\alpha}{y}=-\pdv{\beta}{x}
\end{gather}

Как мы вводили потенциал и функцию тока, у нас было:
\begin{gather}
	\label{eq3}
	v_x=\pdv{\phi}{x}=\pdv{\Psi}{y}\\
	v_y=\pdv{\phi}{y}=-\pdv{\Psi}{x}
\end{gather}
Если взять комплексную функцию $F(z)=\phi+i\Psi$ (действительная часть потенциал, мнимая --  функция тока), то это и будет т.н. комплексный потенциал.

Мы помним, что
\begin{gather}
	\nabla^2\phi=0,\\
	\nabla^2\Psi=0,\\
	\phi=\Re{F},\\
	\Psi=\Im{F},\\
\end{gather}
Или
\begin{gather}
	\label{eq444}
	\phi_1=\Im{F}\\
	\Psi_1=\Re{F}
\end{gather}

\begin{equation}
	v=|F'_z|, \quad
	v=\sqrt{v_x^2+v_y^2}
\end{equation}
Конкретно можно подставлять одну из двух формул для фи и пси.

Замечание два. Конформное отображение.=====

\begin{gather}
	\label{eqdd}
	z=f(\xi)
\end{gather}
z, xi -- комплексные. Соответственно
\begin{equation}
	F(z)=F(f(\xi))
\end{equation}

Задачу о цилиндре мы будем решать с помощью конформных отображений: крыло самолета в сечении конформными преобразованиями связано с окружностью. Имея решение об обтекании цилиндра, сможем решить задачу об обтекании крыла самолета.

Примеры.

Пример 1a. Однородный поступательный поток. $F(z)=ax+i\cdot ay$, $\phi=ax$, $\Psi=ay$. Что такое линии тока? Это линии $\Psi=\const$. Скорость $v_x=\pdv{\Psi}{x}=a$

---> - линии тока
		%B 
% -----_*->
 % -|--/|->
 % -|-/-|-> v_x
 % -|/--|-> 
% *_----|->
% A
% РИС 1

Задается вопрос на экзамене: у вас есть плоский поток, и вот такая линия по косинусу:

\begin{equation}
	Q=\int\limits_{A}^{B} \vec{v}\vec{n}dl=\Psi(B)-\Psi(A)=a(y_b-y_a)
\end{equation}

%--\
% 	\
% 	 \
% 	  \
% 	  |
% 	  |
% 	 /
% 	/
%--/
% Линия по косинусу

Пример 1б. $F(z)=iF(z)=ia-a$. Это сопряженное течение:

% -|-|-|->
% -|-|-|->
% -|-|-|->
% Сопряженное течение

Дз. Что будут представлять из себя линии тока, если $\alpha=1,\beta=1$? Нужно найти угол наклона.

Пример 2a. У нас было поле конденсатора, поле точечного заряда. В 2д будет потенциал нити. Чтоб мы могли сами догадаться, название примера - <<сток-исток-вихрь>>.

\begin{equation}
	F(z)=m\ln(z)
\end{equation}
\begin{comment}
     |y
   ______*  
  /     /\
 /     /  \
/     / t  \
-----*--------->x
\          /
 \        /
  \______/
\end{comment}
\begin{gather}
	\label{eqss}
	x=r\cos\theta\\
	y=r\sin\theta\\
	z=x+iy=re^{i\theta}\\
	F(z)=m\ln(re^{i\Theta})=m\ln(r)+mi\theta
\end{gather}
\begin{comment}
Принцип действия логарифмической линейки
 ______________________________________
|_.__.__.___.___.____._____.____.______| ln x
	 ______________________________________
	|_.__.__.___.___.____._____.____.______| шкала произведения
 ______________________________________
|_.__.__.___.___.____._____.____.______| ln y

x*y
x+y

lnx+lny=ln(xy)
\end{comment}

\begin{equation}
	\phi=m\ln{r}, \quad \Psi=m\Theta, \quad v_r=\pdv{\varphi}{r}=\frac{m}{r}
\end{equation}

Линии тока $\Psi=\const$:

\begin{comment}
	Тут короче кружочек с радальными стрелочками из центра
\end{comment}

Посчитаем поток. 
\begin{equation}
	Q=\int\limits_{A}^{B} \vec{v}\vec{n}dl=\Psi(B)-\Psi(A)=m(\Theta_b-\Theta_a)
\end{equation}

Давайте честно посчитаем интеграл:
\begin{equation}
	Q=\int\limits_{A}^{B} \vec{v}_r\vec{r}dl=.... % лень писать. Говорят что надо доверять здравому смыслу
\end{equation}

Делаем следующую вещь:
\begin{comment}
окружность, кривулька два раза по окружности:
   ______
  //----\\
 //      \\
//        \\
*          ||
\          //
 \        //
  \______//
  лень
\end{comment}

Проблема была в том, что контур охватывал особую точку $r=0$, И ВСЕ ЭТО НЕ РАБОТАЕТ

Пример 2б. Сопряженное течение.
\begin{equation}
	F(z)=im\ln{z}=im\ln{re^{i \theta}}=im\ln{r}-m\theta
\end{equation}
\begin{equation}
	\phi=m\theta, \quad
	\Psi=m\ln{r}
\end{equation}
Тогда
\begin{equation}
	v_r=\pdv{\phi}{r}=0, \quad
	v_\theta=\frac{1}{r}\pdv{\Psi}{\theta}=\frac{m}{r}
\end{equation}

\begin{comment}
Линии Psi=const   
   ______
  /      \
 /  ._.   \
/  /   \   \
|  |   |   |
\  \._./   /
 \        /
  \______/
\end{comment}

Циркуляция:
\begin{equation}
	\Gamma=\int \vec{v}d\vec{l}=\int \frac{m}{r}\cdot 2\pi r=2\pi m = \const.
\end{equation}
Течение у нас потенциальное. А у потенциального течения циркуляция ноль! Но опять, мы же захватили особую точку в середине, и получили фигню.

Что сделать самостоятельно (кто сделает сам, 4, кто придумает что сделать, 5): СДЕЛАТЬ СЛУЧАЙ $a=\alpha+i\beta$.

$m$ мнимое -- вихрь, действительное -- сток/исток, а нужно сделать для случая $m$ комплексное.

$\alpha>0$ исток (и вихрь), $\alpha<0$ сток (и вихрь). Вихрь из-за беты.

Подставить и решить, как мы делали ранее:
\begin{equation}
	(\alpha+i\beta)\ln{re^{i\theta}}=
		...
\end{equation}

Вообще задача качественно решается в уме. Сходящаяся или расходящаяся спираль.

Пример 3. Гидродинамический диполь.
\begin{equation}
	F(z)=-\frac{P}{z}=-\frac{P}{x+iy}=\phi+i\Psi
\end{equation}

\begin{comment}
	картинка диполя:
     __>___
 \	/ _>__ \ /
  \	|/    \|/
----*      *-------
  / |\_>__/|\
 /	\__>___/ \
\end{comment}

Найдите фи и пси:
\begin{equation}
	\phi=-\frac{Px}{x^2+y^2}, \quad
	\Psi=\frac{Py}{x^2+y^2}
\end{equation}

Линии тока $x^2+y^2=c\cdot y$ -- смещенные окружности.
\begin{equation}
	x^2+y^2-2\frac{1}{2}cy+\frac{1}{4}c^2=\frac{1}{4}c^2 \quad
	x^2+(y-c/2)^2=r^2
\end{equation}
Отсюда $r=\frac{c}{2}$, центр окружности в $(0,c/2)$. Теперь действительно:
\begin{comment}
		   _<_
		  /	  \
		 /	   \
		 \	   /
		  \	o /
------------*----------------->
		 	o
		  /	  \
		 /	   \
		 \	   /
		  \_<_/
\end{comment}

Надо вспомнить, что $\phi=-\frac{Px}{x^2+y^2}$, а $v_x=\pdv{\phi}{x}$. Ну и надо найти скорость, и надо немножко быть ленивым. Чтобы не брать производную от этой функции. Можно взять игрек 0, да и посчитать производную в точке. Ведь линия тока знак везде имеет одинаковый.

Пример 4. Циркуляционное обтекание кругового цилиндра. У нас имеется 
\begin{comment}
   ______
  /	     /\
 /	   R/  \
/	   /    \
*     *     |  <----------поток набегает на 
\           /
 \         /
  \_______/
\end{comment}
\begin{equation}
	v_r=0, r=R, \quad
	v+x=-v_0, r\to\infty.
\end{equation}

Все просто?
\begin{equation}
	F=F_1+F_2
\end{equation}
$F_1=-v_0z$ -- просто набегает поток
$F_2=-\frac{v_0}{z}-\frac{A}{z}=-v_0r e^{i\theta} -\frac{A}{r} e^{-i\theta}$

Считаем $v_r$:
\begin{equation}
	\pdv{\phi}{r}=-\qty(v_0-\frac{A}{r})\cos\theta
\end{equation}
здесь при $r=R$ $v_r=0$, и мы нашли константу $A=v_0R^2$.
Тогда
\begin{equation}
	F=-v_0\qty(z+\frac{R^2}{z}).
\end{equation}

\begin{comment}
	   <--\
           \
   ______   \
  /	     /\  \
 /	   R/  \  \
/	   /    \  \_____
*     *     |  <----------поток набегает на 
\           /
 \         /
  \_______/
\end{comment}

\begin{equation}
	F=-v_0\qty(z+\frac{R^2}{z})-\frac{\Gamma i}{2\pi}\ln{z}
\end{equation}

Что такое здесь гамма, во-первых? Что такое гамма здесь? Что? Посмотрим немножко назад, мы считали циркуляцию по контуру. $\Gamma=m 2\pi$ Вопрос. Граничным условиям удовлетворяет? Мы конструируем решение (в силу линейности уравнения Лапласа), поэтому надо проверять. Проверяем граничные условия. Первые два слагаемых только что проверили. А добавили еще вот такое круговое течение (машет руками по кругу). Добавили условие, что нормальная компонента не изменилась.

Ну давайте найдем, что у нас получится.
\begin{equation}
	v_r=\pdv{\phi}{r}=-v_0\qty(1-\frac{R^2}{r})\cos\theta
\end{equation}
\begin{equation}
	v_\theta\bigg|_{r=R}=2v_0\sin\theta+\Gamma\over{2\pi R}
\end{equation}

Смотрим на поверхность. Что на ней делается. На ней первое слагаемое: у нас обтекает поток у нас обтекает поток, почему скорость в два раза больше, почему скорость в два раза больше? Понятно почему, понятно почему. Почему я задумался, а вот знак у меня правильный или неправильный? Но это не суть, важно что должен быть минус, проверьте на экзамене, ответ должен быть правильный, а не то что я написал.

Смотрите, цилиндр
\begin{comment}
  2v_0 <--\
           \
   ______   \
  /	     /\  \
 /	   R/  \  \
/	   /    \  \_____
*     *     |  <---------- v_0
\           /
 \         /
  \_______/
\end{comment}
Если мы что-то перекрыли, поток на поверхности больше. Плюсом еще добавилось круговое течение. Что есть? Есть поток, это вот такая картиночка, это одно решение:
\begin{comment}
       <--\
           \
   ______   \
  /	     /\  \
 /	   R/  \  \
/	   /    \  \_____
*     *     |  <---------- v_0
\           /
 \         /
  \_______/
  (обтекание)
\end{comment}
другое:
\begin{comment}
  /-----<--\
 /          \
/    ______  \
|   /	   \  \  
|  /	    \  \   
| /	         \  |
| *     *     | |
| \           / |
\  \         / /
 \  \_______/ /
  \__________/  
  (по кругу)
\end{comment}
Следующая картиночка вот какая-то такая
\begin{comment}
	картинка с критической траекторией
\end{comment}
Критическое значение
\begin{equation}
	\Gamma^*=4\Pi v_0 R.
\end{equation}
Если гамма меньше критического, то критическая точка на поверхности цилиндра, если больше, то вне области цилиндра. Идет какая-то сепаратриса:
\begin{comment}
	картинка с сепаратрисой. Студент Иванов построил картинки. Линии уровня.
\end{comment}

Вопрос. Переходим к формуле Жуковского. Что будет делаться с цилиндром? Появится сила или нет, действующая на цилиндр? Появится какая-нибудь сила или нет? Чем надо воспользоваться:
\begin{equation}
	p_0+\frac{\rho v_0^2}{2}=p_s+\frac{\rho v_\theta^2}{2}
\end{equation}
Отсюда
\begin{equation}
	p_s=p_0+\frac{\rho}{2}\qty(v_0^2-v_\theta^2)
\end{equation}
Ну и получается
\begin{equation}
	p_s=p_0+\frac12\rho\qty(v_0^2-4v_0^2\sin^2\theta-\qty(\frac{\Gamma}{2\pi R})^2-\frac{2\Gamma v_0\sin\theta}{\pi R})
\end{equation}
Тогда
\begin{equation}
	F=-\int p\vec{n}d\vec{l}.
\end{equation}

По горизонтали силы никакой нет. Давление больше где скорость меньше, значит оно больше внизу -> возникнет подъемная сила. (скорость больше, сложение скорости набегающего потока и кругового).

Чтобы по вертикали посчитать силу, нужно пять слагаемых интегрировать. Не ноль только последнее слагаемое:
\begin{equation}
	F_x=-\int p n_x \dd{l}=\rho \Gamma v_0. 
	% Место на всякий случай
\end{equation}
Это есть формула Жуковского. Интеграл тривиален. Сила пропорциональна плотности, скорости и завихреванности.

Для упражнения вам такой вопрос: у меня есть цилиндр, скатывающийся по наклонной плоскости. (картинка). Одна траектория в вакууме, другая в воздухе. Объяснить какая дальше.

Следующий вопрос. ЧМ по футболу, угловой, мячик, мы ударяем по мячику. Объяснить как бить для сухого листа.

% лекция 22.03.2019
 
Параграф 2.8. Вихревые движения в идеальной жидкости. До сих пор мы занимались потенциальными течениями: сток исток, двумерное движение и т.д.

Вот введем такую величину (вихрь, завихренность):
\begin{equation}
	\vec\Omega=\Rot \vec{v}
\end{equation}
Ранее мы вводили 
\begin{equation}
	\Gamma=\oint \vec{v} \,\dd\vec{l}
	% по теореме стокса
	=\int_S \Rot \vec{v}\dd \vec{S}=\int_S \vec\Omega\vec{n}\,\dd S 
\end{equation}
Циркуляция скорости по замкнутому контуру равна потоку завихренности.

(рис)

Теорема Томсона: циркуляция скорости по замкнутому контуру, двигающемуся в идеальной жидкости, постоянна.

Сохраняется постоянным и поток вихря через поверхность, натянутую на контур, $\Gamma=\const$.

Опр. Вихревая линия - линия, касательная к которой в каждой точке коллинеарна вектору вихря.

Понятие вихревой трубки. Берем некий первоначальный контур $S$, вдоль каждой точки границы проводим вихревую линию - это будет вихревая трубка. (рис)

Теорема Лагранжа. Элементы идеальной жидкости, лишенные вихрей в начальный момент времени, будут лишены их в дальнейшем.

Вихри в идеальной жидкости возникнуть и исчезнуть не могут. Чтобы он появился, необходимо взаимодействие с поверхностью, что приводит к неодносвязности контура (мы к этому вернемся), а также непотенциальность сил. Пример непотенциальной силы: заряженная жидкость в магнитном поле.

Посмотрим теорему Лагранжа более аккуратно. 

(рис)

На поверхности трубки разместим мааленький контур $S_1$. Сосчитаем поток через этот маленький контур: очевидно, он равен нулю 
\begin{equation}
	\Gamma_1=\int \vec\Omega \vec{n}\dd S_1 = 0
\end{equation}

Теперь, если он равен нулю в некий начальный момент, то и в любой момент этот контур будет находится на поверхности этой трубки: так как контур произвольный, представим мысленно трехмерную картинку, что мы берем за кончик контур и немного оттягиваем от трубки. Тогда сразу поток сразу станет не равен нулю. Поскольку контур произвольный, то это означает, что любое малое шевеление с уходом от поверхности приводит к появлению потока. Это доказательство от противного. Это была первая теорема Гельмгольца, о том, что любая линия состоит из одних и тех же элементов жидкости.

Теорема Гельмгольца: поток вектора вихря через поперечное сечение лучевой трубки остается постоянным. 

Боковая поверхность лучевой трубки $S_b$

\begin{equation}
 	\int \vec \Omega \vec{n} \dd S=\int_{S_1} \vec \Omega \vec{n} \dd S  + \int_{S_2} \vec \Omega \vec{n} \dd S + \int_{S_b} \vec \Omega \vec{n} \dd S
 \end{equation} 

Очевидно, поток через боковую поверхность равен нулю. Используем формулу Остроградского-Гаусса: интеграл от омеги по поверхности равен интегралу дивергенции омеги по дв:
\begin{equation}
	\int \vec \Omega \vec{n} dS=\iiint \Div\vec \Omega \dd V=0
\end{equation}
Получится
\begin{equation}
	\int_{S_1} \vec \Omega \vec{n} \dd S=\int_{S_2} \vec \Omega \vec{n} \dd S
\end{equation}
Значит, поток постоянный.

Если $\Omega$ можно считать константой, $S$ мало, то $\Omega S$ -- интенсивность вихревой трубки -- ?

\begin{equation}
	\Div \vec{v}=0, \quad
	\vec \Omega=\Rot\vec{v}, \quad
	\pdv{\vec \Omega}{\tau}=\Rot[\vec{v},\vec{\Omega}\,]
\end{equation}

Пример 1) Плоское течение (рис) $\vec{v}=(v_x(y),0,0)$. Дивергенция очевидно равна нулю:
\begin{equation}
	\Div\vec{v}=\pdv{v_x}{x}+\pdv{v_y}{y}=0
\end{equation}

Сосчитаем завихренность $\vec\Omega$.

\begin{equation}
	\vec{\Omega}=\mqty| % It's require package "physics"
		\vec{i}&\vec{j}&\vec{k}\\
		\pdv{x}&\pdv{u}&\pdv{z}\\
		v_x&0&0
	|=-\vec{k}\cdot\pdv{v_x}{y}
\end{equation}

Итак, завихренность есть, но она направлена либо на нас, либо от нас по рисунку.

Является ли данное течение стационарным? Когда у нас была гидростатика, мы находили условие равновесия, а потом проверили, будет ли это состояние стационарным. Проверка третьим уравнением, будет.

а) $v_x=b\, y \quad \vec{\Omega}=-b\vec{k}$
б)
\begin{equation}
	v_x(y)=\left\{
	\begin{aligned}
		v_0&, \quad y>0\\
		0&, \quad y<0\\
	\end{aligned}
	\right.
\end{equation}
Это ступенчатая функция:
\begin{equation}
	\vec{\Omega}=\vec{k}\cdot v_0\delta(y)
\end{equation}
Это так называемая вихревая пелена. (рис)

Сосчитаем интеграл туПО по контуру:
\begin{equation}
	\Gamma=lv_0+0+0+0
\end{equation}
С другой стороны,
\begin{equation}
	\Gamma=\int \vec{\Omega}\vec{n}\dd S = lv_0
\end{equation}
Что здесь немного непривычного? Мы привыкли считать, что завихренность  --- это когда что-то крутится. А у нас течение плоско-параллельное, но скорости у разных сечений разные: и у нас движение все равно вихревое.

Пример 2) Вращение жидкого цилиндра. (две картиночки)

% Цилиндр соосный z, 3d, ось вверх.
\begin{equation}
	v_\theta(r)=\left\{
	\begin{aligned}
		\omega r&, \quad r<R\\
		0&, \quad r>R\\
	\end{aligned}
	\right.
\end{equation}

Здесь можно обойтись формулой попроще. Вектор Омеги направлен по оси $z$, и получается простая формула: 
\begin{equation}
	\Rot\vec{v}=
		\vec\Omega=
		\vec{I}_z \frac{1}{r}\pdv{r}\,rv_\theta=
		\vec{I}_z 2\omega.
\end{equation}
(график линейный в тета от эр)
$r>0$, течение потенциально: $\vec{\Omega}=0$. Считаем, что цилиндр крутится, а течение вокруг него потенциальное.
\begin{equation}
 	\pdv{r} rv_\theta=0 \quad \Rightarrow \quad v_\theta=\frac{A}{r}
 \end{equation} 
Решаем
\begin{equation}
	\Gamma=\int_L \vec{v}\dd\vec{l}=2\pi R\, v_\theta= 2\pi A
\end{equation}
С другой стороны, гамма это поток вихря:
\begin{equation}
	\Gamma=\iint \Rot\vec{v}\vec{n}\dd S =\iint  \Omega_n \dd S = 2\omega\, \pi R^2
\end{equation}
Из этих формул находим
\begin{equation}
	v_\theta=\frac{\omega R^2}{r}=\frac{\Gamma}{2\pi r}
\end{equation}
Ради чего мы все это делали? Надо найти давление:
\begin{equation}
	p+\frac{\rho v^2}{2}=p_0 \quad \Rightarrow \quad
	p=p_0-\frac{\rho v^2}{2}
\end{equation}
Используем уравнение Бернулли, и ищем ошибку, которая здесь была допущена. По формуле получилось на бесконечности и в центре скорость ноль, и давление в центре нуля и на бесконечности одно и тоже. А из опыта известно, что в центре смерча, например, давление пониженное.

Подсказка: использовали уравнение Бернулли, а оно сформулировано в разных случаях: 1) для стационарного потенциального течения во всем пространстве, 2) вдоль лучевой трубки, 3) нестационарное

При $r>R$ можно пользоваться, течение потенциальное. А внутри-то там есть вихрь, течение не потенциально, и формулой Бернулли пользоваться нельзя.

Непонятно как получается такая формула:
\begin{equation}
	r>R: p=p_0-\frac{\rho\omega^2 R^4}{2r^2}
\end{equation}

Внутри пользуемся уравнением Эйлера. Как мы его выводили: масса умножить на ускорение равно силе.
\begin{equation}
	\pdv{v}{t}+\vec{v}\,\nabla\vec{v}=-\frac{\nabla p}{\rho}
\end{equation}
У нас $\pdv{v}{t}=0$. Нужно знать $\vec{v}\,\nabla\vec{v}$ в сферической системе координат. Есть два способа: первый - посмотреть, второй - смотрим на картиночку: частичка движется по окружности, радиус у неё известен $R$, скорость известна $\omega r$. Ускорение центростремительное $a_r=-\frac{v^2}{R}=-\omega^2 r$. Тогда можем записать:
\begin{equation}
	-\omega^2 r  = - \dv{p}{r}\frac{1}{\rho}?
\end{equation}
\begin{equation}
	p(r)=\left\{
	\begin{aligned}
		&p_0-\frac{\rho\omega^2 R^4}{2r^2}&, \quad& r>R\\
		&p_0-\frac{\rho\omega^2 R^4}{2r^2}+\frac{\rho\omega^2 r^2}{2}, \quad& r<R\\
	\end{aligned}
	\right.
\end{equation}
(рис)
Получается, что давление в центре маленькое. Чем больше радиус вихря, тем меньше давление.


Если кто-то хочет подумать. Вихрь имеет такую \/ структуру. Попробуйте показать, что в нем есть подъемная сила.


Займемся следующим приближением.
\subsubsection{Точечные вихри}

Устремляем сечение нашей вихревой трубки к нулю так:
\begin{equation}
	R\to0, \quad \omega\to\infty, \quad \Gamma=2\pi \omega R^2 =\omega A, \quad v_\theta=\frac{\Gamma}{2\pi r}
\end{equation}

Пусть у нас есть много вихрей (рис). Как они между собой будут взаимодействовать? У нас есть уравнение Лапласа, которое работает всюду, кроме вихревых линий. Скорость в данной точке равняется суперпозиции скорости от всех вихрей. Дальше, по теореме Гельмгольца, завихренность переносится частицами жидкости. То есть, скорость точечного вихря $\Gamma_i$ равняется скорости жидкости в данной точке, создаваемой всеми остальными вихрями. 

\begin{equation}
	\dv{\vec{r}_i}{t}=\sum\limits_{k\ne i} \vec{v}_k (\vec{r}_i)
\end{equation}

(чисто техническая картиночка)

\begin{comment}	
	y
	^
	|
y_i	|- - - - * Г_i
	|   .  ` '
	|._______'_______> x
	         x_i
\end{comment}
\begin{equation}
	r=\sqrt{ (x_k-x_i)^2+(y_k-y_i)^2 }, \quad
	\sin\theta=\frac{y_i-y_k}{r}, \quad v_\theta=\frac{\Gamma_k}{2\pi r_{ik}}
\end{equation}
Первое:
\begin{equation}
	\dv{x_i}{t}=-\frac{1}{2\pi}\sum\limits_{k\ne i} \frac{\Gamma_k (y_i-y_k)}{r^2_{ik}}
\end{equation}
\begin{equation}
	\dv{y_i}{t}=-\frac{1}{2\pi}\sum\limits_{k\ne i} \frac{\Gamma_k (x_i-x_k)}{r^2_{ik}}
\end{equation}
Какие интегралы есть у этой системы?
\begin{equation}
	\sum \Gamma_i\dv{x_i}{t}=-\sum\sum \frac{\Gamma_k\Gamma_i (y_i-y_k)}{r^2_{ik}}=0
\end{equation}
Это значит, что
\begin{equation}
	\sum x_i \Gamma_i=\const=\overline{x}\sum \Gamma_i, \qq{где} \overline{x}=\frac{\sum x_i \Gamma_i}{\sum \Gamma_i}
\end{equation}
Тривиально, что то же самое можно записать по координате $y$. Если $n>2$, других интегралов нету.

Если есть желающие, попробуйте составить программу, которая бы решала эту систему уравнений. Мышкой ставить точку, какая интенсивность гаммы i вводится, мышкой вторая точка ставится интенсивность, и т.д. каждая точка своим цветом, старт, картинки крутятся и вертятся.

Если есть два вихря противоположных знаков, то они уходят на бесконечность: нужно динамическое изменение области просмотра.


Рассмотрим более простую задачу. Всего два вихря: $r_1$ и $r_2$.
\begin{equation}
	\dv{x_1}{t}=-\frac{\Gamma_2 (y_1-y_2)}{2\pi r^2}, \quad r^2=\sqrt{(x_1-x_2)^2+(y_1-y_2)^2}.
\end{equation}
\begin{equation}
	\dv{x_2}{t}=\frac{\Gamma_1 (y_1-y_2)}{2\pi r^2}, \quad r^2=\sqrt{(x_1-x_2)^2+(y_1-y_2)^2}.
\end{equation}
\begin{equation}
	\dv{y_1}{t}=\frac{\Gamma_2 (x_1-x_2)}{2\pi r^2}, \quad r^2=\sqrt{(x_1-x_2)^2+(y_1-y_2)^2}.
\end{equation}
\begin{equation}
	\dv{y_2}{t}=-\frac{\Gamma_1 (x_1-x_2)}{2\pi r^2}, \quad r^2=\sqrt{(x_1-x_2)^2+(y_1-y_2)^2}.
\end{equation}
Домножаем, складываем:
\begin{equation}
	\Gamma_1 x_1 + \Gamma_2 x_2=\const, \quad
	\Gamma_1 y_1 + \Gamma_2 y_2=\const
\end{equation}
Мы нашли два интеграла: центр тяжести не изменяется.
Ну, давайте попробуем еще найти интегралы. Если мы найдем еще два интеграла, то задачу сделаем. Подсказка: из первого уравнения вычесть второе.
\begin{gather}
	\dv{t}\qty(x_1-x_2)=-\frac{\qty(\Gamma_1+ \Gamma_2)}{2\pi}\frac{y_1-y_2}{r},\\
	\dv{t}\qty(y_1-y_2)=\frac{\qty(\Gamma_1+ \Gamma_2)}{2\pi}\frac{x_1-x_2}{r}
\end{gather}
Домножаем, складываем:
\begin{equation}
	(x_1-x_2)\dv{x_1-x_2}{t}+(y_1-y_2)\dv{y_1-y_2}{t}=0 \quad \Rightarrow \quad
	\dv{(x_1-x_2)^2}{t}+\dv{(y_1-y_2)^2}{t}=0
\end{equation}
Отсюда еще один интеграл
\begin{equation}
	(x_1-x_2)^2+(y_1-y_2)^2=r^2=\const	
\end{equation}

Вихри вращаются вокруг неподвижного центра тяжести, с сохранением расстояния между ними. Какие при этом траектории движения? Чтобы найти траекторию, из первого уравнения находим $x_1$, из второго $y_2$, подставляем в это уравнение и получим уравнение окружности.

а) Что же у нас теперь будет? Посмотрим частный случай, когда $\Gamma_2=0$, $\Gamma_1=\Gamma$. (рис) Движение по окружости $v_\theta=\frac{\Gamma}{2\pi r}$, центр тяжести в точке ненулевого вихря.

б) Два вихря $\Gamma, \Gamma$ -- центр тяжести посередине, расстояние между вихрями $l$, $v_\theta=\frac{\Gamma}{2\pi l}$.

в) Два вихря $\Gamma, -\Gamma$. Центр тяжести будет находится на бесконечности (нетрудно посчитать). $v=\frac{\Gamma}{2\pi l}$. Два таких вихря двигаются с постоянной скоростью.

г) Гамма над плоскостью. (рис). Метод изображений даст, что вихри будут двигаться параллельно плоскости.

д) Вихрь в угле - как-то параллельно? Нужен (рис).



% Лекция 29.03.
На прошлой лекции для совокупности вихрей:
\begin{gather}
	\label{eqga}
	\dv{\vec{\Gamma_i}}{t}=\sum v_k(\Gamma_i), \quad v_k=\frac{\Gamma_k}{2\pi r_{ik}}
\end{gather}


Параграф: поверхностные гравитационные волны. (Карабельные волны, цунами, ветровые волны, внутренние волны в неоднородной жидкости)

Приближения, которые мы используем:

1) идеальная, несжимаемая (нет вязкости, звука)
2) однородная (плотность постоянна)
3) поверхность жидкости плоская и неограниченная (земля в данных масштабах плоская)
4) волны малой амплитуды.

(картинка)
\begin{comment}

       |<------ l ----->|
   ________        ___________
  /        \       /          ^
/           \ _ _ /           | a
------------------------------------------
\end{comment}
\begin{gather}
	\label{eq}
	\pdv{\vec{v}}{t}+(\vec{v},\nabla)\vec{v}=-\frac{\nabla p}{\rho}+\vec{g}
\end{gather}
Оценка:
\begin{gather}
	\label{eq}
	\pdv{v}{t}\sim\frac{V}{\tau}, \quad a\sim V\tau, \quad \tau \sim \frac{a}{V},
	\quad \pdv{v}{t}\sim \frac{v^2}{a}, \quad +(\vec{v},\nabla)\vec{v}\sim \frac{v^2}{l},
	\quad l\gg a
\end{gather}
Значит вторым слагаемым можно пренебречь:
\begin{gather}
	\label{eq}
	\pdv{\vec{v}}{t}=-\frac{\nabla p}{\rho}-\nabla(gz)
\end{gather}
Берем от этого уравнения ротор:
\begin{equation}
	\Rot\Grad=0, \quad
	\Rot\vec{v}=0, \quad \vec{v}=\Grad\phi,
\end{equation}
Из несжимаемости
\begin{equation}
	\Div\vec{v}=0
\end{equation}
Тогда
\begin{equation}
	\nabla^2\phi=0
\end{equation}.
Волны описываются (нет силы тяжести, плотности воды) уравнением Лапласа. 
Давайте подумаем, откуда же что делается. 
Давайте напишем граничные условия.

(рис)

Начнем с поверхности. На поверхности воздух, есть некое давление $p_0$. Можем здесь использовать уравнение Бернулли: вернутся к уравнению Эйлера, вместо скорости вставить градиент фи, получим
\begin{equation}
	\rho\pdv{\phi}{t}+\rho g\xi=-\rho p_0
\end{equation}
Введем смещение волны от плоскости $\xi(x,y)$.

Первое приближение: всегда можем ввести $\phi'=p_0\rho t+\phi$
Второе приближение: возмущения достаточно малы, и получится
\begin{equation}
	\rho\pdv{\phi}{t}+\rho g\xi=0 \quad \bigg|_{z=\xi\approx0}
\end{equation}

Посмотрим, чему равняется вертикальная скорость:
\begin{equation}
	v_z=\dv{\xi}{t}=\pdv{\xi}{t}+\xcancel{(v_z\Grad)\xi}
\end{equation}
Из-за малости колебаний вторым слагаемым пренебрегли:
\begin{equation}
	v_z=\dv{\xi}{t}=\pdv{\xi}{t}
\end{equation}
С другой стороны, 
\begin{equation}
	v_z=\pdv{\xi}{t}=\pdv{\phi}{z}
\end{equation}
Дифнем по тэ уравнение для фи:
\begin{equation}
	\pdv[2]{\phi}{t}+g\pdv{\phi}{z}=0\bigg|_{z=\xi\approx0}
\end{equation}
Еще нужно граничное условие на дне -- условие непротекания -- вертикальная компонента скорости на дне равна нулю:
\begin{equation}
	\pdv{\phi}{z}=0 \bigg|_{z=-H}
\end{equation}
Если $z\to\infty$, то $\phi \to 0$.

Введем замену $\phi=\Phi(z)\cdot e^{i(kx- \omega t)}$. Тогда
\begin{equation}
	\dv[2]{\Phi}{z}-k^2\Phi=0
\end{equation}
Оно взялось из $\nabla^2\phi=0$. Нам надо найти такое решение, чтобы оно удовлетворяло нулю на дне. 
Решение:
\begin{equation}
	\Phi=A\ch{k(z+H)}
\end{equation}
Тогда
\begin{equation}
	\phi=A\ch{k(z+H)}\cdot e^{i(kx- \omega t)}
\end{equation}
Считаем производные:
\begin{equation}
	\pdv{\phi}{z}=Ak\sh{k(z+H)}\cdot e^{i(kx- \omega t)}
\end{equation}
\begin{equation}
	\pdv[2]{\phi}{t}=-\omega^2A\ch{k(z+H)}\cdot e^{i(kx- \omega t)}
\end{equation}
В итоге получаем дисперсионное уравнение:
\begin{equation}
	-\omega^2A\ch{k(z+H)}\cdot e^{i(kx- \omega t)}+gAk\sh{k(z+H)}\cdot e^{i(kx- \omega t)}=0 \bigg|_{z=0}
	\quad \Rightarrow \quad
	\omega^2=gk\th{kH}
\end{equation}
Вспоминаем:
\begin{equation}
	\th{x}=\frac{e^x-e^{-x}}{e^{x}+e^{-x}}=\frac{\sh x}{\ch x}
\end{equation}

(график)

У нас есть два масштаба: длина волны $\lambda$ и глубина водоема $H$.

Волны на мелкой воде: $\lambda \gg H$ или $kH\ll 1$.

Получается для волн на мелкой воде
\begin{equation}
	\omega^2=gk^2H, \quad \omega=\pm k\sqrt{gH}
\end{equation}

Волны на мелкой воде -- это волны без дисперсии. Пример яркий волн без дисперсии - это цунами в океане. Глубина океана 5-6 км, а цунами длиной волны десятки, сотни километров.

$v_f=v_{gr}$, если у нас волны на мелкой воде и $v_f=v_{gr}=\sqrt{gH}$.

(оставил парочку строчек)

Второе. $kH \gg 1$. Это волны на глубокой воде. В этом случае тангенс равен 1, и тогда
\begin{equation}
	\omega=\pm\sqrt{gk}
\end{equation}
Волна не чувствует дно, но появляется дисперсия. Примеры волн на глубокой воде -- рябь в луже.

$v_f=\sqrt{\frac{g}{k}}$, $v_{gr}=\frac{g}{2\sqrt{gk}}=\frac12 v_f$.

У нас есть понятие фазовой и групповой скоростей:
\begin{equation}
	v_f=\frac{\omega}{k}=\sqrt{gH}\cdot\sqrt{\frac{\th{kH}}{kH}}, \quad
	v_{gr}=\dv{\omega}{k}=
\end{equation}

Если у нас есть некий волновой пакет, и мы смотрим как он распространяется, у нас есть огибающая и есть фаза. Групповая скорость - скорость огибающей, фазовая - скорость постоянной фазы.

Нарисуем графики, как выглядит фазовая скорость.

(график) 

\begin{comment}
v_f
^              _
| \           / 
|  \       /
|   \    /
|    \_/
|
|________________________________>l
\end{comment}

На мелкой воде пакет волн двигается вместе, а на глубокой гребни волн будут убегать вперед. 

*) Задание.
Оценить время расплывания из второй производной
\begin{equation}
	\pdv[2]{\omega}{k} 	
\end{equation} 


Нас интересует, как двигаются частицы. Запишем 
\begin{equation}
	\phi=A\ch{k(z+H)}\cdot\exp(i(kx-\omega t))
\end{equation}

Частичка находится на некотором горизонте. У нас 
\begin{equation}
	v_x=\pdv{\phi}{x}, \quad v_z=\pdv{\phi}{z}
\end{equation}
(потом посчитаем)

Дальше у нас есть координаты: 
\begin{equation}
	\dv{\xi}{t}=v_x, \quad \dv{\eta}{t}=v_z
\end{equation}

В результате мы сможем определить траектории, по которым двигается частица.

Лекция 05.04.2019
-------------------

на дне условие не протекания, наверху нестационарное уравнение Бернулли используем. В результате мы решили уравнение, использовав граничные условия, и получили дисперсионное соотношение.

\begin{equation}
    \phi = A \ch{k(z+H)}\exp(i(kx-\omega t)), \quad
    v = \nabla \phi
\end{equation}
Мы находимся на некотором горизонте $z$:
\begin{equation}
    v_x = \pdv{\phi}{x} = A\ch{k(z+H)}e^{i(kx-\omega t)}
\end{equation}
\begin{equation}
    v_z = \pdv{\phi}{z} = k\sh{k(z+H)}e^{i(kx-\omega t)} 
\end{equation}
Отсюда
\begin{gather}
    \dv{\xi}{t} = v_x, \quad \xi=\frac{v_x}{-i\omega} = -\frac{kA}{\omega}\ch{k(z+H)e^{(\ldots)}}\\
    \dv{\eta}{t} = v_z, \quad \eta=\frac{v_z}{-i\omega} = \frac{ikA}{\omega}\sh{k(z+H)e^{(\ldots)}} \\
\end{gather}
При $z = 0$ $\xi = a$ (амплитуда на поверхности равна амплитуде колебаний):
\begin{equation}
    A = \xi|_{z = 0} = \frac{ikA}{\omega}\sh{kH}
\end{equation}
Поскольку величины у нас комплексные, мы должны взять действительную часть и получить такой ответ:
\begin{gather}
    \xi = -\frac{a}{\sh{kH}}\ch{k(z+H)}\sin{kx-\omega t}\\
    \eta = \frac{a}{\sh{kH}}\sh{k(z+H)}\cos{kx-\omega t}\\
\end{gather}
Отсюда видно, что траектории двигаются по эллипсу:
\begin{equation}
    \frac{\xi^2}{a_\xi^2}+\frac{\eta^2}{a_\eta^2} = 1, 
    \quad a_\xi = \frac{a\ch{k(z+H)}}{\sh{kH}}
    , \quad a_\eta = \frac{a\sh{k(z+H)}}{\sh{kH}}
\end{equation}
Надо рассмотреть два случая.


\subsubsection{Волны на мелкой воде ($kH\ll 1$)}
Разложение гиперболических функций при малых аргументах:
\begin{equation}
    \sh{x}\approx x+\ldots, \quad \ch{x}\approx 1+\ldots
\end{equation}
Учтем ещё, что по картинке $k(z+H)$ еще меньше, чем $kH$. Тогда
\begin{equation}
    a_\xi = \frac{a}{kH}, \quad
    a_\eta = a\qty{1+\frac{z}{H}}
\end{equation}

(картинка с эллипсом и волной)

Частички двигаются по сильно вытянутым траекториям

\subsubsection{Волны на глубокой воде. Волны с сильной дисперсией $kH\gg 1$}
По определению, 
\begin{equation}
    \sh{x} = \frac{e^x+e^{-x}}{2}, \quad \ch{x} = \frac{e^x-e^{-x}}{2}
\end{equation}
Тогда 
\begin{equation}
    \sh{kH}\approx\ch{kH}\approx \frac{e^{kH}}{2} 
\end{equation}
В общем, получится
\begin{equation}
    a_\xi = a_\eta = ae^{kZ}, \quad z<0
\end{equation}
Траектории представляют собой окружности, быстро спадающие с глубиной. 
Займёмся численным экспериментом. 
Оценка такая: мы находимся в море, амплитуда волны $a$ равна пяти метрам.
Это очень серьёзные волны. 
Давайте считать, что длина волны $\lambda$ тоже десять метров, а мы опустились на глубину 10 метров.
Какая будет амплитуда колебания? Если посчитать, то будет один сантиметр. Вот так быстро спадает. Наверху шторм, а на глубине фактически стоит штиль.

Резюме.
Длинные волны -- это волны без дисперсии, траектории продолговатые эллипсы. $v_f = v_{gr} = \sqrt{gH}$. 
\begin{equation}
    \pdv{\xi}{t}+v\pdv{\xi}{x} = 0 ???, \xi = \xi_0(x-vt)
\end{equation}
Выйдем за пределы и будем считать, что
\begin{equation}
    v = \sqrt{g(H+\xi)}
\end{equation}
Попробуйте представить себе, что будет. Волна подходит к берегу, и возникает два эффекта: за счёт малой $H$, энергия то никуда не девается, амплитуда становится больше. Второй эффект в возрастании скорости гребня (быстрее чем подошва), и волны опрокидывается. 

Короткие волны -- это волны с дисперсией, и их траектории -- окружности. При этом $v_f = \frac{\omega}{k} = \ldots$, $v_{gr} = \dv{\omega}{k}$

\subsection{Гравитационно-капиллярные волны}
Ранее мы не учитывали поверхностное напряжение. Нарисуем картинку

(картинка)

Есть два главных радиуса кривизны, и возникает избыточное давление:
\begin{equation}
    \delta P = \alpha\qty(\frac{1}{R_1}+\frac{1}{R_2}) = \alpha \frac{1}{R} = -\alpha\nabla \qty(\frac{\nabla\xi}{\sqrt{1+(\nabla\xi)^2}})
\end{equation}
Если $\nabla\xi \sim \frac{a}{\lambda} \ll 1$, то $\delta P_k = -\alpha \Delta \xi$.

Задача свелась к предыдущей. Граничное условие на дне -- не протекание, а на поверхности другое:
\begin{equation}
    \rho\pdv{\phi}{t}+\rho_r g\xi+p_0+\delta P_k = 0, \quad z = 0
\end{equation}
(градиент выкинули в силу линейности задачи, считаем колебания малыми)

От $p_o$ мы легко избавляемся, и в итоге получаем уравнение:
\begin{equation}
    \pdv{\phi}{t}+g\xi+\alpha\Delta\xi = 0 \quad\bigg|_{z = 0}
\end{equation}
Что дальше с этим делать? То же самое, что и раньше. У нас встретились фи и кис, надо от кис избавиться. Вспоминаем:
\begin{equation}
    \Delta\xi = \pdv[2]{\phi}{x}
\end{equation}
??? 
В результате получаем граничное условие:
\begin{equation}
    \pdv[2]{\phi}{t}+g\pdv{\phi}{z}-\frac{\alpha}{\rho}\frac{\partial^3\phi}{\partial z^2 \partial t} = 0
\end{equation}
Решаем уравнение Лапласа, учитываем граничные условия на дне, и получается следующее решение для потенциала. Теперь подставляем это решение в последнее уравнение, и получаем дисперсионное уравнение
\begin{equation}
    \omega^2 = (gK+\gamma k^3)\th{kH}, \quad \gamma = \frac{\alpha}{\rho}.
\end{equation}
\begin{equation}
    kH \gg 1 \quad \Rightarrow \quad \omega^2 = gK+\gamma k^3.
\end{equation}
Когда существенно капиллярные волны? Для ряби на воде, понятно что для цунами они не важны. Давайте считать фазовую скорость:
\begin{equation}
    v_f^2 = \frac{\omega^2}{k^2} = \frac{g}{k}+\gamma k 
\end{equation}
Рисуем график

(график с минимумом) $\lambda_0 = 1.7 \text{ см}$

Ищем, чему равен минимум: он равен $k_{*} = \sqrt{\frac{g}{k}}$

\begin{equation}
    v_{gr} = \dv{\omega}{k} \quad \Rightarrow \quad 
    v_{gr} = \frac{v_f}{2}\frac{k_{*}^2+3k^2}{k_{*}^2+k^2}
\end{equation}

При очень маленьких $k$ $v_{gr} = \frac{v_f}{2}$ (как на мелкой воде). 
При больших же $v_{gr} = \frac{3}{2}v_f$.
Итак, у нас есть капиллярные и гравитационные волны.
В одном случае групповая меньше фазовой, в другой больше фазовой. Баобабовому

Резюме:
Гравитационные и капиллярные волны.

Дисперсионное уравнение:
\begin{equation}
    \omega^2 = (gk+\gamma k^3)\th{kH}
\end{equation}
Если $k \gg k_*$, это капиллярные волны. 
Если $\frac{1}{H} \ll k \ll k_*$, то это гравитационные короткие волны (дно ещё не чувствуется).
Если же $k \ll \frac{1}{H}$, то это длинные гравитационные волны.

Попробуем обсудить вот такую задачу:

(картинка) $\rho_0 \gg \rho_1$.

Здесь мы возвращаемся к гидростатике.
Является ли такое состояние решением уравнения гидростатики? (вверху тяжелея, внизу лёгкая).
Гравитация параллельна градиенту плотности, это хорошо. 
А является ли такое состояние устойчивым? 

Нужно написать уравнение Лапласа, поставить граничные условия, написать дисперсионное уравнение с учётом капиллярных сил. 
Считаем, что поверхность достаточно недалеко.

%надо переставить одну спичку
Задача отличается от предыдущей чем?
Перевернули вверх ногами.
Тогда уравнение дисперсии
\begin{equation}
    \omega = \sqrt{-gk+\gamma k^3}
\end{equation}

(картиночка)

Тут возникают проблемы: отрицательные $\omega^2$. $k_* = \sqrt{\frac{g}{k}}$. Если у вас ка больше ка со звёздочкой (мелкие возмущения), то всё нормально:
\begin{equation}
    \omega_{1,2} = \pm\sqrt{\gamma k^3}
    \quad \Rightarrow \quad
    \xi = c_1e^{i\omega_1 t}+c_2e^{i\omega_2 t}
\end{equation}
А при мнимых омега:
\begin{equation}
    \xi = c_1e^{-|\omega|t}+c_2e^{|\omega|t}
\end{equation}

Итак, при малых возмущениях жидкость просто колеблется,а при больших начинает течь.

Значит, решение устойчивое.
Жизненный пример: перевёрнутый стакан с водой.
Более жизненный -- духи.
Если духи поставить, то хорошие выливаться не будут.
Потому что горлышко узкое, и масштабы маленькие, а ка большое (подавлены крупномасштабные возмущения).
Надо их потрясти: согласно Эйнштейну, движение с ускорением эквивалентно увеличению силы тяжести:
\begin{equation}
    k_{*\text{э}} = \sqrt{\frac{g+a}{k}}
\end{equation}

Следующий раздел -- внутренние волны

\subsection{Внутренние волны}
Задать малые возмущения, лианеризовать и т.д. Частота Брента-Вейсаля.

На следующей лекции рассмотрим простую задачу двухслойной жидкости $\rho_1, \rho_2$.
Ось зет вверх, зададим возмущение на поверхности и надо решить эту задачу.

Решение ищем в виде бегущей волны, считаем что вверх и вниз убывает (экспоненциально).
Граничные условия: на границе одинаковая скорость ($\dv{\phi_1}{z} = \dv{\phi_2}{z}$).
Ещё одно граничное условие -- давление на поверхности: сверху и снизу границы давление одинаково.

В итоге напишем дисперсионное уравнение. В пределом случае получим гравитационные волны (при $\rho_1 \to 0$).

% 12.04

Итак, у нас такая задача
Считаем, что до границы достаточно далеко.
На границе есть какие-то возмущения
(рисунок)
\begin{equation}
    \Delta\phi_1 = 0, \quad \Delta\phi_2 = 0
\end{equation}
Справа налево начнём писать уравнение.
Будем искать решение в виде
\begin{gather}
    \phi_1 = A e^{-kz} e^{i(kx-\omega t)}\\
    \phi_2 = B e^{kz} e^{i(kx-\omega t)}
\end{gather}
\begin{equation}
    \phi_1 = \Phi(z)\cdot e^{i(kx-\omega t)}
\end{equation}

Наша задача -- найти дисперсионное уравнение. Займёмся константами. Мы по-прежнему считаем, что колебания относительно малы,
т.е. амплитуда колебаний много меньше длины волны.

Уравнение непрерывности в таком виде (совпадение скоростей на границе):
\begin{equation}
    v_z = (\approx) \pdv{\xi}{t} = \pdv{\phi_1}{z} = \pdv{\phi_2}{z}
\end{equation}
Отсюда сразу следует $B = -A$.

Мы должны поставить второе граничное условие. Используем нестационарное уравнение Бернулли:
\begin{equation}
    p_1 = -\rho_1 g \xi - \rho_1 \pdv{\phi_1}{t}, \quad
    p_2 = -\rho_2 g \xi - \rho_2 \pdv{\phi_2}{t}
\end{equation}
Здесь мы пренебрегли слагаемым $v^2$ в силу малости колебаний.
Граничные условия -- давления на поверхности одинаковы:
\begin{equation}
    kg(\rho_2-\rho_1) = \omega^2(\rho_2+\rho_1)
\end{equation}
Мы избавились от $\xi$ и в итоге получили дисперсионное уравнение:
\begin{equation}
    \omega = \sqrt{gk\frac{\rho_2-\rho_1}{\rho_2+\rho_1}} \approxeq \sqrt{gK\frac{\Delta \rho}{\rho}}
\end{equation}
Предельный случай: $\rho_1 \to 0$ даёт переход к случаю гравитационных волн на глубокой воде. 
В океане $\Delta \frac{\rho}{\rho}\sim 10^{-2}$.

Явление мертвой воды, обнаруженное Нансеном, заключается в том, что судно на поверхности пресной воды толщиной примерно с судно на соленой воде, тратит всю энергию не на передвижение, а на создание внутренних волн.

Переход к частоте БВ
\begin{equation}
    \pdv{\vec{v}}{t}+(\vec{v}\,\nabla)\vec{v} = -\frac{\nabla p}{\rho}+\vec{g}\\
    \pdv{\rho}{t}+\Div \rho \vec{v} = 0
\end{equation}

Уравнение гидростатики:
\begin{equation}
    \dv{p_0}{z} = -g \rho_0(z)
\end{equation}

\begin{equation}
    \pdv{\vec{v}}{t} = -\frac{\nabla p_0+\nabla p_1}{\rho_0+\rho'}+\vec{g}
\end{equation}

Несжимаемость даст
\begin{equation}
    \dv{\rho}{t} = 0 \quad \Rightarrow \quad
    \Div \vec{v} = 0
\end{equation}

Нетривиальность в следующем: не хватает одного уравнения: переменных 5, уравнений скалярных 4.

Вспомним полную производную:
\begin{equation}
    \pdv{\rho}{t}+\Div \rho \vec{v} = 0
\end{equation}
\begin{equation}
    \pdv{\rho}{t}+(\vec{v}\,\nabla)\rho = 0
\end{equation}
Это записано для возмущений. $\rho = \rho_0+\rho'$:
\begin{equation}
    \pdv{\rho_0}{t}+(\vec{v}\,\nabla)\rho = 0
\end{equation}
\begin{equation}
    (\vec{v}\,\nabla(\rho_0+\rho'))
\end{equation}
Пренебрежём $\vec{v}\,\nabla \rho'$, тогда получим ещё уравнение
\begin{equation}
    \pdv{\rho'}{t}+v_z\dv{\rho_0}{z} = 0
\end{equation}

Вспомним и запишем частоту Брента-Вяйсаля:
\begin{equation}
    N^2 = -g\dv{\rho_0}{z}\frac{1}{\rho_0}
\end{equation}
Тогда последнее уравнение перепишем в виде
\begin{equation}
    \pdv{\rho'}{t}-v_z \frac{N^2}{g}p_0 = 0
\end{equation}
Приближение Неслышу--неска.

Без всякого вывода рассмотрим один частный случай: экспоненциальная атмосфера, где $\rho_0(z) = \rho_0 e^{-\frac{z}{H}}$, и $N^2 = gH$. 
\begin{equation}
    v_z = A\exp{-i\omega t+ik_xx+ik_zz}
\end{equation}
Здесь $H$ -- эффективная высота атмосферы. Дисперсионное уравнение здесь будет
\begin{equation}
    \omega^2 = N^2\cdot k_x^2
\end{equation}
Можно записать так:
\begin{equation}
    \omega = N\sin\theta, \qq{где} \sin{\theta} = \frac{k_x}{k}
\end{equation}
1) Волны существуют только с частотой $\omega<N$.
2) Зависимость направления от частоты. Если $\omega \to N$, то волновой вектор направлен горизонтально. Если же $\omega \ll N$, напротив, вертикально.
3) (рисуночек) $\vec{v_f} \perp \vec{v}_{gr}$.

Рисунок с профилем звука в волноводном канале. Из закона сохранения нужно найти закон спадания.
\begin{equation}
   p^2\cdot 2\pi r l \quad \Rightarrow \quad p \sim \frac{1}{r} 
\end{equation}

Попробуем решить такую задачу. В районе Англии произошла катастрофа: взорвались торпеды у подводной лодки. Во сколько раз будет больше акустическое давление в волноводном канале по сравнению с сферической волны? $H \sim 100$ м, мы на побережье Америки. $\sqrt{\frac{l = 5000}{H = 0.1}} \approx 7000$.

\section{Движение в вязкой несжимаемой жидкости}

Мы уже столкнулись с рядом парадоксов: парадокс Даламбера -- на тело в потоке жидкости не действует сила. 
Ясно,что это всё не так. Второе -- в жидкости не могут образоваться вихри. Третье -- у нас было граничное условие непротекания:
равенство нулю нормальной компоненты скорости. Но на поверхности неподвижного тела просто скорость равна нулю (опытный факт). 
Парадокс: собирается пыль на поверхности вентилятора, хотя он крутится. Это следствие равенства нулю скорости. 

\subsection{Уравнения гидродинамики вязкой жидкости} %par 3.1
Останется ранее выведенное уравнение. Мы его получили без предположения о не вязкости:
\begin{equation}
    \pdv{\rho}{t}+\Div \rho \vec{v} = 0
\end{equation}
Второе уравнение 
\begin{equation}
    \pdv{\vec{v}}{t}+(\vec{v}\,\nabla)\vec{v} = -\frac{\nabla p}{\rho}+\vec{g}
\end{equation}

Уравнение Эйлера мы вывести смогли, а вот для жидкости вязкой его вывести нельзя. Мы его просто сконструируем.
Рассмотрим такую экспериментальную задачу: у нас есть пластинка на высоте $h$ над нижней (нижняя бесконечная). 
К пластинке площадью $S$ прикладываем силу $F$, и она двигается со скоростью $v$.
Опыт показывает, что чтобы пластинка двигалась с постоянной скоростью, нужно
\begin{equation}
    \frac{F}{S} = \eta \frac{V_0}{h}
\end{equation}
$\eta$ -- это динамический коэффициент вязкости. Для жидкости он убывает с ростом температуры, а для газов медленно растёт.
Можно ввести ещё один коэффициент:
\begin{equation}
    \nu = \frac{\eta}{\rho}.
\end{equation}


%type    eta nu
%Вода 0.01 0.01
%Воздух 1.8\cdot10^{-4}  0.15
%Спирт 0.018    0.022
%Глицерин 8.5   6.8
%Ртуть 0.0156   0.0012
Если мы хотим что-то разогнать, где будет больше? В воздухе.
Будем постепенно переходить к уравнениям. Мысленно выберем в жидкости небольшую площадочку: $\Delta x, \Delta y, \Delta z$.

(picture)

По аналогии можем записать:
\begin{equation}
    \frac{\Delta F}{\Delta S} = \eta \frac{\Delta v_x}{\Delta x}
\end{equation}
Попытаемся сконструировать (но не вывести!) уравнения движения вязкой жидкости.
За основу возьмём закон сохранения импульса:
\begin{equation}
    \pdv{t}\rho v_i = -\pdv{\Pi_{ik}}{x_k}
\end{equation}
Он фундаментален и верен и для вязкой жидкости. А вот ЗСЭ не выполняется: есть трение.

\begin{equation}
    \Pi_{ik} = p \delta_{ik}+\rho v_i v_k+\sigma_{ik}
\end{equation}

Добавили одно слагаемое -- тензор вязких напряжений $\sigma_{ik}$. 
Попробуем его собрать на основе логичных предположений.

Жидкость двигается как целое -- силы трения нет. Она возникает только при смещении слоёв: это первое предположение.
Второе -- мы не рассматриваем экстремальные движения, и считаем что эта зависимость ($\sim \eta$) линейна.
Это линеаризация. Вязкие силы возникают на молекулярных масштабах.
Жидкость будем считать изотропной. Наиболее общий вид тензора вязких напряжений:
\begin{equation}
    \sigma_{ik} = a\qty(
        \pdv{v_i}{x_k}+\pdv{v_k}{x_i}
    )+
    c\qty(
       \pdv{v_i}{x_k}-\pdv{v_k}{x_i}
    )+
    b\sum \pdv{v_i}{x_k} \delta_{ik}
\end{equation}
Если жидкость двигается как целое, силы трения нет. В каких ещё случаях её нет?
Когда жидкость крутится как целое. Это вот что такое:
\begin{equation}
    \vec{v} = [\vec{\Omega}\times \vec{r}\,]  =
    \mqty|
    \vec{i} & \vec{j} & \vec{k}\\
    \Omega_x & \Omega_y & \Omega_z\\
    x& y & z
    |
\end{equation}
Отсюда
\begin{equation}
    v_x = \Omega_y z - \Omega_z y
\end{equation}
(pass two lines -- for exam)
\begin{equation}
    v_y = \Omega_x z - \Omega_z x, \quad v_z = \Omega_x y - \Omega_y x
\end{equation}

Если честно подставить  в (), получим
\begin{equation}
    \pdv{v_i}{x_k}+\pdv{v_k}{x_i} = 0
\end{equation}
\begin{equation}
    \sum \pdv{v_i}{x_k} \delta_{ik} = 0
\end{equation}

Тогда слагаемое 
\begin{equation}
    \pdv{v_i}{x_k}-\pdv{v_k}{x_i}
\end{equation}
Нулю уже не равно, но при вращении как целого трения нет, значит тензор равен нулю. Значит, коэффициент $c = 0$.

\begin{equation}
    \sum \pdv{v_i}{x_k} \delta_{ik} = 
    \pdv{v_1}{x_1}+\pdv{v_2}{x_2}+\pdv{v_3}{x_3} = \Div \vec{v}
\end{equation}
Оно существенно, когда жидкость сжимаемая. Резюме:
\begin{equation}
    \sigma_{ik}= \eta\qty(
        \pdv{v_i}{x_k}+\pdv{v_k}{x_i}
    )+ \xi\sum \pdv{v_i}{x_k} \delta_{ik}
\end{equation}

Программа следующей лекции: у нас есть уравнение, выражение для тензора $\Pi_{ik}$, и т.п. Должны получить уравнения Навье-Стокса.
Система дифференциальных уравнений в частных производных движения вязкой ньютоновской жидкости.

%Лекция от 19.04.2019

Мы взяли за основу уравнение (потому что есть закон сохранения импульса, который выполняется и для вязкой среды):
\begin{equation}
    \pdv{t}\rho v_i = -\pdv{\Pi_{ik}}{x_k}-\sigma_ik, \quad 
    \Pi_{ik} = p\delta_{ik} + \rho v_i v_k
\end{equation}
Импульс меняется за счёт сил. Первое слагаемое -- за счёт давления, второе -- .
Ещё добавили тензор вязких напряжений $\sigma$. В результате, предполагая что жидкость изотропна
и матрица из 9 элементов симметрична, получаем три константы. После предположения о вращении
жидкости как целого, избавились ещё от одной константы и получили
\begin{equation}
    \sigma_{ik} = \eta \qty(\pdv{v_i}{x_k}+\pdv{v_k}{x_i})+\xi \sum \pdv{v_l}{x_l} v_{k}
\end{equation}
Теперь запишем (используя уравнение непрерывности $\pdv{\rho}{t} = -\pdv{\rho v_k}{x_k}$)
\begin{equation}
    \pdv{\rho}{t}v_i + \rho \pdv{v_i}{t} =  \pdv{t} \rho v_i = \ldots
\end{equation}
Надо суметь продифференцировать эти (?) слагаемые. Это элементарная работа, далее вместо производной плотности 
задействуем уравнение непрерывности.
\begin{equation}
    \pdv{x_k} \rho v_i v_k = v_i \pdv{x_k} \rho v_k + v_k \rho \pdv{v_i}{x_k}
\end{equation}
Замечание:
\begin{equation}
    \pdv{p\delta_{ik}}{x_k} = \pdv{p_k}{x_k}
\end{equation}
В итоге получим
\begin{equation}
    \rho\qty(
        \pdv{v_i}{t}+v_k\pdv{v_i}{x_k}
    )=
    -\pdv{p}{x_i}+\eta\pdv[2]{v_i}{x_k}+\qty(\frac{\eta}{3}+\xi)\pdv[2]{v_k}{x_i x_k}
\end{equation}
Немножко сгруппировали слагаемые. Это и есть уравнение Навье Стокса, его можно записать в векторной форме:
\begin{equation}
    \rho\qty(\pdv{\vec{v}}{t}+(\vec{v}\,\nabla)\vec{v}) = -\nabla p + \eta \Delta \vec{v} +
   \qty(\frac{\eta}{3}+\xi)\Grad\Div \vec{v}
\end{equation}

Можно ввести коэффициент $\frac{\eta}{\rho} = \nu$, тогда
\begin{equation}
    \pdv{\vec{v}}{t}+(\vec{v}\,\nabla)\vec{v} = -\frac{\nabla p}{\rho}+\nu \Delta \vec{v}
\end{equation}
Размерность 
\begin{equation}
    [\eta] = \frac{L^2}{T}
\end{equation}

Итак, мы получили уравнения Навье-Стокса.

\subsubsection{Тензор вязких напряжений несжимаемой жидкости}

\begin{equation}
    E_k = \frac{\rho}{2}\int v^2 \dd{V}
\end{equation}
Можно показать, что 
\begin{equation}
    \dv{E_k}{t} = \frac{\rho}{2} \int 2\pdv{v_i}{t} v_i \dd{V}=
    -\frac{\eta}{2} \int   \qty(\pdv{v_i}{x_k}+\pdv{v_k}{x_i})^2 \dd{V}
\end{equation}

Как определять коэффициенты $a, b$, которые мы вводили раньше? Мы конструировали уравнения, не выводя.
Их надо как-то определять экспериментально.

Можно взять круглый диск (касающийся жидкости) на нити, закрутить и определить добротность осциллятора с затуханием. Можно бросить шарик в сосуд. Или прокачивать жидкость через трубу, и скорость вытекания будет как-то зависеть от вязкости.

Рассмотрим такую задачу:

\begin{figure}[h!]
    \centering
    \includegraphics[]{example-image-a}
    \caption{Caption here}
    \label{fig:figure1}
\end{figure}

Верхняя поверхность жёсткая... Рассмотрим стационарный случай:
\begin{equation}
    \pdv{t} = 0
\end{equation}
Кроме того,
\begin{equation}
    \pdv{\Pi_{ik}}{x_k} = 0,\quad
    \Pi_{ik} = p\delta_{ik}+\rho v_i v_k -\sigma_{ik}
\end{equation}
Проинтегрируем это уравнение по внутреннему объёму:
\begin{equation}
    \int_S \Pi_{ik} \eta_k \dd{S} + \int_{S'} \Pi_{ik} \eta_k \dd{S} = 0
\end{equation}
На внутренней поверхности, если жидкость идеальная, то равна нулю нормальная компонента: для вязкой же жидкости равен нулю модуль скорости. Тогда
\begin{equation}
    F_i = -\int p \eta_i \dd{S} - \underbrace{\int \nabla_{ik} \eta_k \dd{S}}_{F'_i}
\end{equation}
У нас есть две силы: первая -- обычное давление, и вязкая сила на единицу поверхности:
\begin{equation}
    f'_i = \sigma_{ik} n_k
\end{equation}
Дурацкий вопрос: скорость равна нулю, а откуда возникает сила? А из равенства нулю скорости не следует равенство нулю градиента.

Будем использовать достаточно часто.

В принципе, мы можем найти силу, с которой жидкость действует на тело, измеряя силу $F'$.

Если каким-то способом удалось измерить скорости и градиенты вдали от тела, то можно по ним найти, чему равна сила.

Должно быть очевидно, что
\begin{equation}
    F_i = F'_i
\end{equation}
(не успел написать почему)

Замечание о граничных условиях.

Твердое неподвижное тело: скорость на поверхности равна нулю.

\begin{figure}[h!]
    \centering
    \includegraphics[]{example-image-a}
    \caption{Caption here}
    \label{fig:figure1}
\end{figure}
\begin{equation}
    \vec{v}= (v_x (y),0,0)
\end{equation}
Наверху жидкости никакой нет, поэтому
\begin{equation}
    f_i = \sigma_{ik} n_k =
    \eta\pdv{v_x}{y} = 0
\end{equation}

\subsubsection{Плоское течение между двумя пластинками (течение Куэтта)}
\begin{figure}[h!]
    \centering
    \includegraphics[]{example-image-a}
    \caption{Caption here}
    \label{fig:figure1}
\end{figure}

Запишем уравнение Навье-Стокса:
\begin{equation}
    \rho\qty(
        \pdv{v_i}{t}+v_k\pdv{v_i}{x_k}
    )=
    -\pdv{p}{x_i}+\eta\pdv[2]{v_i}{x_k}+f_i
\end{equation}
Начнём упрощать:
\begin{equation}
    \pdv{t} = 0,\quad
    \vec{v} = \qty(v_x (y),0,0), \quad
    p = p(y)
\end{equation}
В этих предположениях 
\begin{equation}
    -\pdv{p}{y} = -g, \quad
    p = p_0 + \rho_0 g y
\end{equation}
Запишем уравнение в проекции на ось ..
\begin{equation}
    \eta\pdv[2]{v_x}{y} = \pdv{p}{y} = 0
\end{equation}
И здесь граничные условия
\begin{equation}
    v_x (0) = 0, \quad
    v_y (H) = V_0
\end{equation}
Будем искать решение в виде $v_x (y) = cy$, тогда
\begin{equation}
    cH = V_0 \quad \Rightarrow \quad c= \frac{V_0}{H}
\end{equation}

Все это было ради силы, мы хотели найти именно её.
\begin{equation}
    f_i = \sigma_{ik} n_k 
\end{equation}
В нашем случае $n_k \equiv n_y = -1$, в таком случае
\begin{equation}
    f_y = -\eta \pdv{v_x}{y}= -\eta \frac{V_0}{H}
\end{equation}
Сила, которая действует на площадку. Умножив её на площадь, получим первую экспериментальную формулу, которую мы получили. Если
бы выводили строго, вместо $\eta$ писали бы некую константу, и
только в конце узнали бы что из эксперимента это $\eta$.

\subsubsection{Течение Пуазейля}
\begin{figure}[h!]
    \centering
    \includegraphics[]{example-image-a}
    \caption{Caption here}
    \label{fig:figure1}
\end{figure}
Будем рассматривать стационарное течение. Граничное условие
\begin{equation}
    v = v(r), \quad v(R) = 0
\end{equation}
На поверхности трубы у нас скорость равняется нулю. Дальше, 
опять же, возвращаемся к уравнению Навье-Навье в векторной форме:
\begin{equation}
    \pdv{\vec{v}}{t}+(\vec{v}\,\nabla)\vec{v} = 
    - \frac{\nabla p}{\rho}+\nu \Delta \vec{v}
\end{equation}
В левой части стоит ускорение. У нас частички все двигаются по прямой с постоянной скоростью, значит, скорость и градиент будут со направлены и скалярное произведение ноль. Значит, вся левая часть обращается в нуль. Считаем, что внешних сил никаких нет, тогда получаем уравнение (вспомним, что $\nu = \frac{\eta}{\rho}$
\begin{equation}
    \pdv{p}{z} = \eta 
   \qty[
   \frac{1}{r} \pdv{r} r \pdv{v}{r}
   ]
\end{equation}
Отсюда первое тривиальное заключение
\begin{equation}
    \pdv{p}{z} = c_1
\end{equation}
Тогда
\begin{gather}
    \eta\frac{1}{r} \pdv{r} r \pdv{v}{r} = a_1
    \quad \Rightarrow \quad
    \eta r\pdv{v}{r} = c_1 \frac{r^2}{2} + c_{1}\\
    v = \qty(\dv{p}{z}) \frac{r^2}{4\eta}+A \ln(r) + B
\end{gather}
Из физической реализуемости $A = 0$, а из условия
\begin{equation}
    v(R) = 0 \quad \Rightarrow \quad
    v(r) = \qty(\dv{p}{z}) \frac{1}{4\eta}\qty(r^2-R^2)
\end{equation}

Если вода бежит вправо, то слева давление больше, и градиент скорости отрицательный. Тогда профиль скорости будет таким:

\begin{figure}[h!]
    \centering
    \includegraphics[scale=0.5]{example-image-a}
    \caption{Caption here}
    \label{fig:figure1}
\end{figure}

Сосчитаем поток:
\begin{equation}
    Q=2\pi \int\limits_0^R v(r) r \dd{r} =
    \frac{\pi}{8\eta} \qty(\pdv{p}{z})\cdot R^4
\end{equation}
Там, где самая большая скорость (в центре), элемент площади при интегрировании очень мал. Поэтому получается такая зависимость.

Что будет, если в трубу вставить ещё одну трубу? Останется константа $A$, исчезнет нерегулярность логарифма. Придётся ставить два граничных условия, на обеих трубах.

Ещё может быть, что внутренний диаметр трубы стремится к нулю. Какой будет при этом профиль? Нужно сделать аккуратно предельный переход.

Последний комментарий по этой задаче -- найдём силу:
\begin{equation}
    f_z = \eta_{ik} n_k = -\sigma_{zr} = -\eta \pdv{v_r}{r}=
    -\frac{1}{2} \qty(\dv{p}{z})R
\end{equation}

\subsubsection{Нестационарное движение вязкой жидкости. Вязкие волны}

Постановка задачи такая: у нас гармонически колеблется пластинка, сверху имеется жидкость. Как она будет колебаться?
\begin{figure}[h!]
    \centering
    \includegraphics[scale=0.5]{example-image-a}
    \caption{Caption here}
    \label{fig:figure1}
\end{figure}
Запишем уравнение Навье-Навье для несжимаемой жидкости (из соображений симметрии $\vec{v}=(v_x (z),0,0)$, очевидно):
\begin{gather}
    \Div \vec{v} = 0 \quad \Rightarrow \quad \pdv{v_z}{z} = 0\\
    \pdv{v_x}{dt} = \nu \pdv[2]{v_x}{z}
\end{gather}
Будем искать решение в виде 
\begin{equation}
    v_x = A e^{i\omega t + ikz}
\end{equation}
Тогда
\begin{equation}
    -i \omega = \nu k^2 \quad \Rightarrow \quad
    k = (1+i) \sqrt{\frac{\omega}{2\nu}}
\end{equation}
Можно ввести величину
\begin{equation}
    \delta=\sqrt{\frac{2\nu}{\omega}}
\end{equation}
Это толщина скин-слоя. 
