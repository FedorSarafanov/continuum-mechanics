%!TEX root = ../lections.tex
\subsection{Уравнение Эйлера}
Уравнение движение идеальной жидкости ( аналог \textbf{2 закона Ньютона})
Второй закон Ньютона для жидкого элемента
\begin{align*}
\rho d V \frac{d \vec{v}}{d t} &=\vec{F_{S}}+\vec{F} \\
\vec { F } &= \rho \vec { f }
\end{align*}
Здесь $F$ объемная сила действующая на элемент $dV$ ( $f$ – сила отнесенная к единице массы, для силы тяжести $f=g$, где g ускорение свободного падения).

$F_s$ - сила действующая на элемент объема со стороны окружающей среды.  В идеальной среде силы трения нет и единственная сила определяется только силами давления. На элемент поверхности $d\sigma$ действует сила $ P d \vec { \sigma } $ и результирующая сила равна:
\begin{align*}
\vec { F } _ { S } = - \oint \limits_ { S } p d \vec { \sigma } = - \int \limits_ { V } \nabla p d V \approx - \nabla p d V
\end{align*}

В результате получаем \textbf{уравнение Эйлера}
\begin{align*}
\frac { d \vec { v } } { d t } = - \frac { \nabla p } { \rho } + \vec { f }
\frac { \partial \vec { v } } { \partial t } + ( \vec { \vec{v} } \nabla ) \vec { v } = - \frac { \nabla p } { \rho } + \vec { f }
\end{align*}
Здесь мы учли что в уравнение Ньютона входит полная производная. У нас  5 неизвестных – 3 компоненты скорости, давление и плотность. И только 4 уравнения: 3 уравнения Эйлера для трех компонент и уравнение непрерывности.
Нужно еще одно уравнение – уравнение состояния, связывающее давление, плотность и энтропию $S$: $ P = P ( \rho , S ) $ и уравнение для энтропии. Для изоэнтропический жидкости $ \frac { d S } { d t } = \frac { \partial S } { \partial t } + \vec { v } \nabla S = 0 $

Если в начальный момент времени энтропия была одинакова во всем пространстве, то она не будет меняться с течением времени и уравнение состояние принимает вид: $ P = P ( \rho ) $.
В идеальном газе уравнение адиабаты имеет вид(\textbf{уравнение Пуассона}):
\begin{align*}
P &= P _ { 0 } \left( \rho / \rho _ { 0 } \right) ^ { \gamma } \\
\gamma &= c _ { p } / c _ { v }
\end{align*}
где для идеального газа $\gamma = \frac{i+2}{i}$, ($i$ - количество степеней свободы).
Для жидкостей дело хуже. В разных диапазонах давления имеют разные уравнения состояния. Эмпирическая формула для давления $P$, измеряемого в атмосферах: 
\begin{align*}
\frac { P + B } { 1 + B } = \left( \frac { \rho } { \rho _ { 0 } } \right) ^ { \gamma }
\end{align*}
где $B=3000\text{ атм}$, $\gamma = 7$, давление до $10^5$ атмосфер.

Итак, \textbf{система уравнений для идеальной жидкости} принимает вид:
\begin{align*}
& \frac { \partial \vec { v } } { \partial t } + ( \vec { v } \nabla ) \vec { v } = - \frac { \nabla p } { \rho } + \vec { g } \\
& \frac { \partial \rho } { \partial t } + d i v ( \rho \vec { v } ) = 0 \\
& P = P ( \rho )
\end{align*}
\subsection{Законы сохранения энергии и импульса для идеальной жидкости}

Энергия единицы объема – кинетическая $+$ внутренняя
\begin{align*}
E = \frac { 1 } { 2 } \rho v ^ { 2 } + \rho \varepsilon
\end{align*}

Закон сохранения энергии в интегральной форме:
\begin{align*}
\frac { \partial } { \partial t } \int\limits_{ V } \rho \left( \frac { 1 } { 2 } v ^ { 2 } + \varepsilon \right) d V = - \oint\limits_{ S } \rho \left( \frac { 1 } { 2 } v ^ { 2 } + \varepsilon \right) \vec { v } d \vec { \sigma } - \oint _ { S } p \vec { v } d \vec { \sigma }
\end{align*}

Изменение энергии в объеме равно притоку (выносу) энергии в объем через границы $+$ работа внешних сил давления. Энтальпия равна $ W = \rho \varepsilon + P $ из курса термодинамики и общей физики. Получаем закон сохранения в интегральной форме:
\begin{align*}
\frac { \partial } { \partial t } \int \limits_{ V } \rho \left( \frac { 1 } { 2 } v ^ { 2 } + \varepsilon \right) d V = - \oint \limits_{ S } \rho \left( \frac { 1 } { 2 } v ^ { 2 } + W \right) \vec { v } d \vec { \sigma }
\end{align*}
По формуле Стокса переходим в правой части от интегрирования по поверхности к интегрированию по объему:
\begin{align*}
\int \limits_ { V } \left[ \frac { \partial } { \partial t } \rho \left( \frac { 1 } { 2 } v ^ { 2 } + \varepsilon \right) + \Div \left( \rho \left( \frac { 1 } { 2 } v ^ { 2 } + W \right) \vec { v } \right] d V = 0\right.
\end{align*}

Поскольку объем произвольный можно перейти к дифференциальной  форме закона сохранения энергии:
\begin{align*}
& \frac { \partial E } { \partial t } + \Div \vec { N } = 0 \\
& E = \frac { \rho v ^ { 2 } } { 2 } + \rho \varepsilon \\
& \vec { N } = \left[ \frac { \rho v ^ { 2 } } { 2 } + \rho \varepsilon + P \right] \vec { v }
\end{align*}
Здесь $E$ – плотность энергии, $N$ – вектор плотности потока энергии – аналог вектора Пойнтинга в электродинамике. Введён в 1874 году Умовым.

\subsection{Закон сохранения импульса}
Для единицы объема жидкости импульс равен $ \vec { p } = \rho \vec { v } $. Если закон сохранения энергии мы выводили в интегральной форме, то здесь мы будем стартовать с дифференциальных уравнений. Запишем изменения для i-ой компоненты:
\begin{align*}
\frac { \partial } { \partial t } \left( \rho v _ { i } \right) = \rho \frac { \partial v _ { i } } { \partial t } + v _ { i } \frac { \partial \rho } { \partial t }
\end{align*}

Запишем уравнение Эйлера и уравнение непрерывности по компонентам:
\begin{align*}
\frac { \partial v _ { i } } { \partial t } + \sum _ { k = 1 } ^ { 3 } v _ { k } \frac { \partial v _ { i } } { \partial t } &= - \frac { 1 } { \rho } \frac { \partial P } { \partial x _ { i } } + f _ { i } \\
\frac { \partial \rho } { \partial t } + \sum _ { k = 1 } ^ { 3 } \frac { \partial \left( \rho v _ { k } \right) } { \partial x _ { k } } &= 0
\end{align*}
В результате для изменения компоненты импульса имеем:
\begin{align*}
\frac { \partial } { \partial t } \left( \rho v _ { i } \right) = - \frac { \partial } { \partial x _ { k } } \left( P \delta _ { i k } + \rho v _ { i } v _ { k } \right) + \rho f _ { i }
\end{align*}
Здесь по индексу $k$ идет суммирование. Хочетсяя привести это уравнение к дивергентной форме, чтобы получить закон сохранения. Учтем:
\begin{align*}
\rho v _ { k } \frac { \partial v _ { i } } { \partial x _ { k } } + v _ { i } \frac { \partial \left( \rho v _ { k } \right) } { \partial x _ { k } } = \frac { \partial \left( \rho v _ { i } v _ { k } \right) } { \partial x _ { k } }
\end{align*}
Внешние силы приводят к иземенеию импульса. Нужно что-то придумать с давлением:
\begin{align*}
\frac { \partial P } { \partial x _ { i } } = \frac { \partial \left( \delta _ { i k } p \right) } { \partial x _ { k } }
\end{align*}
Здесь $ \delta _ { i k } = 1 , i = k ; \delta _ { i k } = 0 , i \neq k $ - символ Кронекера.

В результате получим:
\begin{align*}
\frac { \partial } { \partial t } \left( \rho v _ { i } \right) = - \frac { \partial } { \partial x _ { k } } \left( P \delta _ { i k } + \rho v _ { i } v _ { k } \right) + \rho f _ { i }
\end{align*}

Введем тензор плотности потока импульса: $ \Pi _ { i k } = P \delta _ { i k } + \rho v _ { i } v _ { k } $. Тогда закон сохранения импульса запишется как:
\begin{align*}
\frac { \partial } { \partial t } \left( \rho v _ { i } \right) = - \frac { \partial \Pi _ { i k } } { \partial x _ { k } } + \rho f _ { i }
\end{align*}

Проинтегрируем последнее равенство по объему:
\begin{align*}
\frac { \partial } { \partial t } \int \limits_{ V } \rho v _ { i } d V = - \int \limits_{ V } \frac { \partial \Pi _ { i k } } { \partial x _ { k } } d V + \int \limits_{ V } \rho f _ { i } d V
\end{align*}
Используя теорему Остроградского-Гаусса для тензора получаем:
\begin{align*}
\frac { \partial } { \partial t } \int \limits_ { V } \rho v _ { i } d V = - \oint \limits_ { S } \Pi _ { i k } n _ { k } d \sigma + \int \limits_ { V } \rho f _ { i } d V
\end{align*}
Таким образом, изменение импульса в объеме $V$ связано с потоком импульса через поверхность $S$. Векторная же форма закона сохранения импульса имеем вид:
\begin{align*}
\frac { \partial } { \partial t } \int \limits_ { V } \rho  \vec{v} d V  = - \oint \limits_ { S } [ P \vec { n } + \rho \vec { v } ( \vec { v } \vec{n} ) ]d \sigma
\end{align*}
Здесь $\vec{n}$ - внешняя нормаль.

Следствие: Как использовать закон сохранения импульса для нахождения силы действия потока на тело? Если движение стационарно:
\begin{align*}
\oint _ { S } \left[ p n _ { i } + \rho v _ { i } v _ { k } n _ { k } \right] d \sigma = 0
\end{align*}
Отсюда для силы действия потока на тело имеем:
\begin{align*}
F _ { i } = - \oint \limits_ { S } p n _ { i } d \sigma = \oint \limits_ { S } \rho v _ { i } v _ { k } n _ { k } d \sigma
\end{align*}

Пример - изогнутая трубка(душ).
\begin{align*}
\frac { \partial } { \partial t } \int \limits_ { V } \rho  \vec{v} d V  = - \oint \limits_ { S } [ P \vec { n } + \rho \vec { v } ( \vec { v } \vec{n} ) ]d \sigma
\end{align*}
\begin{figure}[H]
	\centering
	\includegraphics[]{example-image-a}
	\caption{картинка изогнутой трубки}
	\label{fig:figure5}
\end{figure}
В одно сечение жидкость втекает, а из другого вытекает.
\subsection{Гидростатика}
Рассмотрим простейший случай когда скорость жидкости равна нулю. Из исходной системы уравнений 
\begin{align*}
& \frac { \partial \vec { v } } { \partial t } + ( \vec { v } \nabla ) \vec { v } = - \frac { \nabla p } { \rho } + \vec { f } \\
& \frac { \partial \rho } { \partial t } + d i v ( \rho \vec { v } ) = 0 \\
& P = P ( \rho )
\end{align*}
следует
\begin{align*}
& \nabla P = \rho \vec { f } \\
& P = P ( \rho )
\end{align*}

Пусть внешняя сила потенциальна
\begin{align*}
& \vec { f } = - \nabla u \\
& \nabla P = - \rho \nabla u
\end{align*}
то есть градиенты давления и сила параллельны.

При какой зависимости плотности от координаты последнее уравнение имеет решение? Применим к последнему уравнению операцию ротора
\begin{align*}
& \operatorname { rot } ( \nabla P ) = 0 \\
& \operatorname { rot } ( - \rho \nabla u ) = - \rho \operatorname { rot } ( \nabla P ) - [ \nabla P \nabla \rho ] \\ 
& [ \nabla P \nabla \rho ] = 0
\end{align*}
Таким образом вектора градиентов плотности и потенциала должны быть параллельны.

\emph{Распределение давления в поле тяжести.}
\begin{align*}
& \nabla P = \rho \vec { g } \\
& P = P ( \rho )
\end{align*}
то есть в поле тяжести стационарное решение существует, если плотность зависит от высоты.

\textbf{Примеры:}
\begin{enumerate}
	\item {\textbf{Жидкость в поле тяжести}(вода). Плотность постоянна. Ось $z$ направлена вниз.
	\begin{align*}
	& \frac { d P } { d z } = \rho _ { 0 } g \\
	& P = P _ { A } + \rho _ { 0 } g z
	\end{align*}
	Давление увеличивается на 1 атмосферу на 10 метрах.}
	\item {\textbf{Изотермическая атмосфера}(идеальный газ с постоянной температурой $T$). Ускорение можно считать постоянным. Ось $z$ направлена вверх.
	\begin{align*}
	& \frac { d P } { d z } = - \rho ( z ) g \\
	& P = \frac { R } { \mu } \frac { m } { V } T \\
	& P = \frac { R } { \mu } \rho T
	\end{align*}
	Здесь $R$ - универсальная газовая постоянная. $\mu$ - молярная масса газа.
	\begin{align*}
	& \frac { R T } { \mu } \frac { d P } { d z } = - \rho ( z ) g \\
	& \rho = \rho _ { 0 } \exp ( - z / h ) , P = P _ { 0 } \exp ( - z / h ) \\
	& h = \frac { R T } { \mu g }
	\end{align*}
	Здесь $h$ - высота атмосферы, величина порядка 8 км, поэтому изменением силы тяжести можно пренебречь.}
	\item {\textbf{Закон Архимеда}. На тело, погруженное в жидкость,  со стороны жидкости действует выталкивающая сила, равная весу жидкости, вытесненную этим телом.
	\begin{align*}
	\nabla P = \rho \vec { g }
	\end{align*}
	Сила со стороны жидкости на элемент поверхности $ d \vec { F } = - p \vec { n } d S $. Здесь $\vec{n}$ - внешняя нормаль. Тогда сила Архимеда равна:
	\begin{align*}
	& \vec { F }_ { A }  = - \oint \limits_ { S } P \vec{n} d S = - \int \limits_ { V } \nabla P d V = - \int \limits_ { V } \rho \vec { g } d V \\
	& \int _ { V } \rho \vec { g } d V = \vec { P } \approx \rho V \vec { g } \\
	& & \vec { F }_ { A } = - \vec { P }
	\end{align*}
	Здесь $\vec{P}$ - вес вытесненнной жидкости. Причем и плотность, и ускорение \textbf{не обязательно постоянны!}
	}
\end{enumerate}
\subsection{Гидростатическое равновесие. Частота Брента — Вяйсяля}
Выясним условия, при  которых состояние равновесия жидкости в поле тяжести будет устойчивым.
Будем считать что плотность зависит от глубины. Ось $z$ направлена вниз. Элементарный  объем   находится на глубине $z$, потом $z+x$.

На тело действуют две силы: сила тяжести и сила Архимеда, и в равновесии они равны  по величине:
\begin{align*}
F _ { g } ( z ) &= g \rho ( z ) V _ { 0 } \\
F _ { A } ( z ) &= - F _ { g } ( z ) = - g \rho ( z ) V _ { 0 }
\end{align*}

Пусть данный объем смещается по вертикали на расстояние $x$. Масса сохраняется и сила тяжести не меняется.  Пусть \textbf{жидкость несжимаема}, тогда объем не меняется. А сила Архимеда изменяется, так как плотность вокруг частицы изменилась. Тогда уравнение  Ньютона для объема запишется как:
\begin{align*}
& m \frac { d ^ { 2 } x } { d x ^ { 2 } } = g \rho ( z ) V _ { 0 } - g \rho ( z + x ) V _ { 0 } \\
& m = \rho ( z ) V _ { 0 }
\end{align*}

Разлагая плотность в ряд, и ограничиваясь линейными членами, получаем:
\begin{align*}
\frac { d ^ { 2 } x } { d t ^ { 2 } } = - \frac { g } { \rho } \frac { d \rho } { d z } x
\end{align*}
Это уравнение гармонического осциллятора:
\begin{align*}
& \frac { d ^ { 2 } x } { d t ^ { 2 } } + N ^ { 2 } x = 0 \\
& N ^ { 2 } = \frac { g } { \rho } \frac { d \rho } { d z } \propto \frac { g } { L }
\end{align*}
Здесь $ N = \left( \frac { g } { \rho } \frac { d \rho } { d z } \right) ^ { 1 / 2 } $ - частота Брента-Вяйсаля.
\begin{enumerate}
	\item {\textbf{Устойчивость жидкости} наблюдается при $ N ^ { 2 } > 0 , \frac { d \rho } { d z } > 0$. Элемент совершает колебания с частотой $N$.}
	\item {\textbf{Неустойчивость жидкости} наблюдается при $ N ^ { 2 } < 0 $. Элемент падает вниз или стремится всплыть.}
\end{enumerate}
\subsection{Уравнение Бернулли}
Получим альтернативную запись уравнения Эйлера в форме Громэка-Лэмба
\begin{align*}
&\frac { \partial \vec { v } } { \partial t } + ( \vec { v } \nabla ) \vec { v } = - \frac { \nabla P } { \rho } + \vec { g } \\
&\vec { g } = g \nabla z
\end{align*}
Ось $z$ направлена вниз. Учтем два равенства из курсов векторного анализа и термодинамики(для равновестных обратимых изобарических процессов).
\begin{align*}
&(\vec {v} \nabla ) \vec { v } = \frac { 1 } { 2 } g r a d \left( v ^ { 2 } \right) - [ \vec { v } , [\nabla , \vec { v } ]] \\
&\frac { \nabla P } { \rho } = \nabla W \\
&\frac { \nabla P } { \rho } = \nabla \frac { P } { \rho } , \quad \rho = const
\end{align*}

Получаем уравнение Эйлера в форме Громэко-Лэмба
\begin{align*}
\frac { \partial \vec { v } } { \partial t } + \operatorname { grad } \left( \frac { v ^ { 2 } } { 2 } + W - g z \right) = [ \vec { v } , [\nabla , \vec { v } ]]
\end{align*}

Рассмотрим частные случаи:
\begin{enumerate}
	\item {\textbf{Движение стационарно} ($\vec{v}=const$)}
	\begin{itemize}
		\item {\textbf{Безвихревое движение}(потенциальное, $\operatorname { rot } \vec{v}=0$).
		Тогда из уравнения Громэко-Лэмба имеем
		\begin{align*}
		& \operatorname { grad } \left( \frac { v ^ { 2 } } { 2 } + W - g z \right) = 0 \\
		& \frac { v ^ { 2 } } { 2 } + W - g z = const
		\end{align*}
		\textbf{Константа сохраняется во всем пространстве.} Если жидкость несжимаема и однородна, то:
		\begin{align*}
		\frac { v ^ { 2 } } { 2 } + \frac { P } { \rho } - g z = const
		\end{align*}
		Это уравнение Бернулли для стационарного \textbf{потенциального} движения однородной несжимаемой жидкости.
		}
		\item {\textbf{Вихревое движение}($\operatorname { rot } \vec{v} \neq 0$)

		Введем понятие линии тока. \textbf{Линия тока} - это линия, касательные к которой в данный момент времени и в каждой точке совпадают с вектором скорости $v$.  Линии тока определяются системой дифференциальных уравнений.
		\begin{figure}[H]
			\centering
			\includegraphics[scale=1]{example-image-c}
			\caption{Линии тока}
			\label{fig:figure6}
		\end{figure}
		\begin{align*}
		\frac { d x } { d v _ { x } } = \frac { d y } { d v _ { y } } = \frac { d z } { d v _ { z } }
		\end{align*}
		Умножим уравнение Эйлера на вектор скорости, то есть спроектируем на линии тока:
		\begin{align*}
		& \vec {v}[ \vec { v } , [\nabla , \vec { v } ]]=0 \\
		& \vec {v} \perp [ \vec { v } , [\nabla , \vec { v } ]]
		\end{align*}
		Используя определение линии тока
		\begin{align*}
		\vec { v } \nabla ( \ldots ) = \frac { d } { d l } ( \ldots ) = 0
		\end{align*}
		получаем тот же закон сохранения
		\begin{align*}
		\frac { v ^ { 2 } } { 2 } + \frac { p } { \rho } - g z = const
		\end{align*}
		Но здесь константа сохраняется только вдоль линии тока, и \textbf{для разных линий тока константы разные!}
		}
		\end{itemize}
	\item {\textbf{Нестационарное вихревое движение}
	\begin{align*}
	& \frac { \partial \vec { v } } { \partial t } \neq 0 \\
	& \operatorname { rot } v = 0
	\end{align*}

	В силу потенциальности $ \vec { v } = \nabla \varphi $ из уравнения Громэко-Лэмба получаем
	\begin{align*}
	\frac { \partial \phi } { \partial t } + \frac { v ^ { 2 } } { 2 } + \frac { p } { \rho } - g z = \mathrm { const }
	\end{align*}
	Этот интеграл носит название \textbf{интеграла Коши.}

	Лучевая трубка тока, трубка образованная множеством линий тока, проходящей через некоторый замкнутый контур.
	\begin{figure}[H]
		\centering
		\includegraphics[scale=1]{example-image-c}
		\caption{Лучевая трубка}
		\label{fig:figure7}
	\end{figure}

	Закон Бернулли это ничто иное как следствие законов сохранения  массы и энергии вдоль некоторой лучевой трубки через 2 сечения входящее $S_1$ и выходящее $S_2$.

	Закон сохранения массы: \textbf{сколько втекает, столько и вытекает.}
	\begin{align*}
	m _ { i } = \rho _ { i } S _ { i } v _ { i } \Delta t , \quad i = 1,2
	\end{align*}
	Изменение энергии за счет вытекания и работы силы тяжести равно работе внешних сил:
	\begin{align*}
	& A _ { i } = p _ { i } S _ { i } v _ { i } \Delta t \\
	& E _ { i } = \frac { v _ { i } ^ { 2 } } { 2 } + u _ { i } + \varepsilon _ { i } \\
	& A _ { 1 } - A _ { 2 } = \Delta m \left( E _ { 2 } - E _ { 1 } \right)
	\end{align*}
	Здесь $u$ и $\varepsilon$ - потенциальная и внутренняя энергия. Рассмотрим случай несжимаемой жидкости. В этом случае внутренняя энергия не меняется, а $ u = - g z $. В результате получим уравнение Бернулли.

	}
\end{enumerate}

Уравнение Бернулли имеет множество приложений:
\begin{enumerate}
	\item {Трубка Пито
	Знаем сечения, измеряем давления – находим скорости
	\begin{align*}
	& \frac { v _ { 1 } ^ { 2 } } { 2 } + \frac { p _ { 1 } } { \rho } = \frac { v _ { 2 } ^ { 2 } } { 2 } + \frac { p _ { 2 } } { \rho } \\
	& S _ { 1 } v _ { 1 } = S _ { 2 } v _ { 2 }
	\end{align*}
	\begin{figure}[H]
		\centering
		\includegraphics[scale=1]{example-image-c}
		\caption{Трубка Пито}
		\label{fig:figure8}
	\end{figure}
	}
	\item {Обтекание двух цилиндров
	Сближение линий тока, увеличение скорости. Возникает притяжение цилиндров.
	\begin{figure}[H]
		\centering
		\includegraphics[scale=1]{example-image-c}
		\caption{схематический вид цилиндров}
		\label{fig:figure9}
	\end{figure}
	}
	\item {Вытекание жидкости из сосуда
	\begin{align*}
	& \sigma < < S \\
	& v = \sqrt { 2 g \left( z _ { 1 } - z _ { 2 } \right) }
	\end{align*}
	\begin{figure}[H]
		\centering
		\includegraphics[scale=1]{example-image-c}
		\caption{Схематичный вид вытекающей жидкости}
		\label{fig:figure10}
	\end{figure}
	}
	\item {Задача Прандля. Косое падение плоской струи на поверхность. Кумулятивные снаряды. Наряду с уравнением Бернулли нужно использовать закон сохранения импульса.
	\begin{figure}[H]
		\centering
		\includegraphics[scale=1]{example-image-c}
		\caption{Наклонное падение струи}
		\label{fig:figure11}
	\end{figure}
	\begin{figure}[H]
		\centering
		\includegraphics[scale=1]{example-image-c}
		\caption{Кумулятивные снаряды}
		\label{fig:figure12}
	\end{figure}
	}
\end{enumerate}